\documentclass{scrarticle}
\usepackage[utf8]{inputenc}
\usepackage[T1]{fontenc}
\usepackage[ngerman]{babel}
\usepackage{amsmath}
\usepackage{amsfonts}
\usepackage{amssymb}
\begin{document}
Den theoretischen Hintergrund der Masterarbeit bilden die Ort- und Rasterzellen des Hippocampus, die für verschiedenste Aufgaben der Orientierung zuständig sind. Das reicht von abstrakten Zuordnungen wie der Höchstgeschwindigkeit zu einem Fahrzeug auf Grundlage der Motorleistung und des Gewichts bis zur klassischen räumlichen Navigation in einer Stadt oder einem Gebäude. Da diese Resultate bereits per Maschinellem Lernen untersucht wurden, soll es in der Arbeit darum gehen, ob die Methoden diese Methode auch dazu verwendet werden können, um Sprache zu verarbeiten, damit so ggfl. Rückschlüsse auf die Orts- und Gitterzellen gezogen werden können. Zu diesem Zweck soll die Theorie der Projektiven Karten und deren mathematischer Formulierung der Successor Representation genutzt werden.

Um dieses Konzept auf Sprachen anzuwenden, werden verschiedene Techniken des Natural Language Processing sowie ein Neuronales Netz verwendet. Erstere dienen hauptsächlich dazu die Trainingsdaten für Letzteres bereitzustellen. Diese bestehen aus aufeinander folgenden Wortpaaren, eines dienst als Eingabe, das andere als Ausgabe. Ziel ist es aus den Wort-für-Wort-Vorhersagen die grammatikalische Struktur der Sprache dahingehend abzuleiten, dass einzelne Wörter/Zustände im Sinne der Predictive Map-Theorie ihren Nachfolger kodieren.

Um dies zu erreichen, werden mehrere Architekturen eines Neuronalen Netzes untersucht, wobei das Hauptaugenmerk auf der Verarbeitung von Büchern, mit denen die Trainingsdaten generiert werden können, da sie plausible Sprachdaten darstellen.
\end{document}