\chapter{Introduction} \label{ch: introduction}
Understanding and therefore modeling the human brain is a challenge as old as science itself. Through the centuries, mankind contemplated and associated the brain as a complex version of the technology they were surrounded by, starting by comparing it with an abacus up to the \gls{hbp}\footnote{https://www.humanbrainproject.eu/en/} founded by the EU, which tries to simulate and build a ``silicon brain''. The first steps toward this venture were taken by Santiago Ramón y Cajal, who was rewarded with the Nobel Prize in $ 1906 $~\cite{Nobelprize06}. Furthermore, his research includes a full description of a nerve cell and further related concepts, for instance that signals are processed mono-directionally. These results lead directly to Rosenblatt's perceptron~\cite{Rosenblatt58P} and from then on to the nowadays modern field of Deep Learning in Computer Science.

Next to large-scale research projects like the \gls{hbp}, there also exist smaller ones tackling different parts of the brain, for example the hippocampus (\secreff{sec: Hippocampus}). It plays a fundamental role in forming new memories, processing emotions as part of the limbic system and in positioning, a result awarded with the Nobel Prize in $ 2014 $ (\secreff{sec: Hippocampus} \& \secreff{sec: predictive map theory})~\cite{Nobelprize14}. This abstract space, in which the navigation happens, is called \cognitiveroom{} (\secreff{sec: Hippocampus}) and \etal{Stachenfeld} provide an adequate mathematical theory, the \gls{sr} (\secreff{sec: SR}), to transfer the concept into a framework to work and do experiments with.

This master thesis tries to extend the application of the \gls{sr}, which is focused on spatial coordination in~\cite{StBoGe17HPM} to learning languages. For this purpose, some techniques of \gls{nlp} are useful, necessary and applied (\secreff{sec: data preparation}). One motivating indication this plan might work out is given by \etal{Stachenfeld}, who showed that their theory doesn't need well structured surroundings like a city but a topological/graphical environment is sufficient to retrieve viable results. Therefore, P. Stöwer did promising first experiments~\cite{Stöwer21MA}, which will be used as a foundation to generalize his framework (\chapreff{ch: framework}) in an attempt to gain more precise results (\chapreff{ch: results}).
