%%%%%%%%%%%%%%%%%%%%%%%%%%%%%%%%%%%%%%%%%%%%%%%%%%%%%%%%%%%%%%%%%%%%%%%%%%%%%%%%%
%
%  Philipp Rost, 1. Juli 2022
%  The hippocampus and language: Word to word prediction in terms of the successor representation
%  Lehrstuhl fuer Mustererkennung, FAU Erlangen-Nuernberg
%
%%%%%%%%%%%%%%%%%%%%%%%%%%%%%%%%%%%%%%%%%%%%%%%%%%%%%%%%%%%%%%%%%%%%%%%%%%%%%%%%%

% Document class for LME theses: lmedoc
% %LANGUAGE
% %CONFIG
%    The option "german" uses german.sty
%    For english papers, use the "english" option
% Possible types of theses:
% bt - Bachelor's thesis
% mt - Master's thesis
% diss - Dissertation
% sa - Student thesis
%\documentclass[german,bt]{lmedoc/lmedoc}
\documentclass[english,mt]{lmedoc/lmedoc}


% % % % % % % % % % % % % % % % % % % % % % % % % % % %
% MEINE ZUSÄTZE % % % % % % % % % % % % % % % % % % % %
\graphicspath{ {Bilder/} }  % Pfad speichern für \includegraphics{...}
% /home/philipp/LaTeX/GlobaleDateien/todonotes.tex % In dieser Datei habe ich meine eigenen todonotes konfiguriert
\usepackage[disable]{todonotes}
\setlength{\marginparwidth}{2.4cm} % Falls die todonotes am Rand abgeschnitten werden, muss man diesen Parameter konfigurieren
\usepackage{wrapfig} % Für Abbildungen, die in Text eingebettet sind
\usepackage{nameref} % Setzt bspw. Name eines Kapitels statt die Zahl
\usepackage[stable]{footmisc} % Fußnoten in Überschriften (klappt ohne nicht so schön)
\usepackage{float} % Für Option „H“ bei „figure“-Umgebungen
\usepackage{multicol}
\usepackage{array}
\usepackage[export]{adjustbox}
%\usepackage{hyperref}

% Befehle, um das Referenzieren einheitlich zu gestalten:
%	- Verwendung: \secref{\ref{LABEL}}
%		- Man könnte auch nur LABEL übergeben, aber dann öffnet sich keine Liste mit den Labels, was ich als störend empfinde
%	- ~ ist ein statisches Leerzeichen, sodass Bezeichnung und Nummer, bspw. am Zeilenende nicht getrennt werden
%	- \figurename setzt einfach "Figure" oder "Abbildung"
\newcommand*{\figref}[1]{\figurename~#1}
\newcommand*{\tabref}[1]{\tablename~#1}
\newcommand*{\secref}[1]{Section~#1}
\newcommand*{\secreff}[1]{Section~\ref{#1}~\nameref{#1}} % TODO ggfl. entfernen
\newcommand*{\equref}[1]{Equation~#1}
\newcommand*{\chapref}[1]{Chapter~#1}
\newcommand*{\chapreff}[1]{Chapter~\ref{#1}~\nameref{#1}} % TODO ggfl entfernen
\newcommand*{\appref}[1]{Appendix~\ref{#1}~\nameref{#1}}

% Befehle für eine einheitliche Schreibweise von Fachbegriffen
\newcommand*{\srmat}{\gls{sr}-matrix}
\newcommand*{\pymupdf}{\texttt{pymupdf}}
\newcommand*{\spacy}{\texttt{spacy}}
\newcommand*{\python}{\texttt{python}}
\newcommand*{\keras}{\texttt{keras}}
\newcommand*{\numpy}{\texttt{numpy}}
\newcommand*{\pyconll}{\texttt{pyconll}}
\newcommand*{\cognitiveroom}{cognitive room}
\newcommand*{\onehot}[1]{$ 1 $-hot-encoded vector#1}
\newcommand*{\postag}[1]{\gls{pos}-tag#1}

\newcommand*{\comment}[1]{}

% Befehle für Zahlen, Breiten und Höhen
\newcommand*{\twocolpicwidth}{0.46\linewidth} % Breite eines Bildes, wenn man mehrere Bilder in zwei Spalten anordnet (TWO COLumn PICture WIDTH)
\newcommand*{\twocolpicheight}{0.365\textheight} % Höhe eines Bildes, wenn man mehrere Bilder in zwei Spalten anordnet (TWO COLumn PICture HEIGHT)
\newcommand*{\threerowpicheight}{0.28\textheight} % Analog zu oben, für 3 × 2-Bilder



%%%%%%%%%%%%%%%%%%
% pdflatex and lualatex are supported
% ++ "Umlaut" support
%    The package "inputenc" can be used to write Umlaute or the german double s
%    directly. You need to use the correct encoding, e.g. latin1.
\usepackage{iftex}
\ifPDFTeX
  \usepackage[utf8]{inputenc}
  \usepackage[T1]{fontenc}
  \usepackage{lmodern}
\else
  \ifXeTeX
     \usepackage{fontspec}
  \else
     \usepackage{luatextra}
  \fi
\fi

%%%%%%%%%%%%%%%%%%
% ++ use \toprule,\midrule und \endrule in your tables, no \hline or vertical
% columns please
\usepackage{booktabs}
% guessing if space is needed
\usepackage{xspace}

% mathstuff
\usepackage{amsmath,amssymb}
\usepackage{mathtools}
\usepackage{bm}

% Color stuff
\usepackage[usenames,dvipsnames,table]{xcolor} % Habe ich auskommentiert, weil es sonst nicht funktioniert
% let's define a dark blue color
\definecolor{faublue}{RGB}{0,51,102}

% defines units
%\usepackage[binary-units,abbreviations]{siunitx}

% allows inline enumerate
\usepackage[inline]{enumitem}
% and sets them to Arabic
\setlist*[enumerate]{label=(\arabic*)}
% remove widows at end or beginning of a page
\usepackage[all]{nowidow}

% great packages to make nice figures/plots/
\usepackage{tikz}
%\usepackage{pgfplots}

% you can even separate table content and layout
%\usepackage{pgfplotstable}
%\pgfplotsset{compat=newest}

% url package, it also breaks at hyphens
\usepackage[hyphens]{url}

% typesetting numbers and units
%\usepackage{siunitx}
%\sisetup{mode=text}% use text mode for numbers

% ++ \url{} better breaking of urls and connect w. hyperref
\usepackage{url}
% ++ Biblatex
%    Replaces the old 'bibtex'. The bibtex step has to be replaced with 'biber'.
\usepackage[backend=biber, bibencoding=utf8, giveninits=true,
maxbibnames=99, % show all authors in the bibliography
maxalphanames=1, minalphanames=1, style=alphabetic,%
style=trad-alpha, backref=true]{biblatex}
% biblatex backref
\DefineBibliographyStrings{english}{%
  backrefpage = {cited on p\onedot},
	backrefpages = {cited on pp\onedot},
}

% ++ use for refercence farther away: \vref
\usepackage{varioref}
\renewcommand\reftextfaraway[1]{(p.\,\pageref{#1})}
% ++ Makes all the references in the document clickable.
%    To ensure that backref is working, this package has to be loaded after biblatex.
\usepackage{hyperref}
\hypersetup{
  colorlinks = true,   % Führt zu einem farbigen Ausdruck!
  linkcolor =  faublue,
  urlcolor =   magenta,
  citecolor =  faublue,
  plainpages =        false,
  hypertexnames =     true,
  linktocpage =       true,
  bookmarksopen =     true,
  bookmarksnumbered = true,
  bookmarksopenlevel= 0,
% pdf information, uncomment if done
  pdftitle =    {The hippocampus and language: Word to word prediction in terms of the successor representation},
  pdfauthor =   {Philipp Rost},
  pdfsubject =  {Master's thesis},
  pdfkeywords = {hippocampus, successor, representation, sr, word, language, prediction, to, word to word, successor representation, neuroscience, cognitive, computational, ccn}
}
% Enable correct jumping to figures when referencing
\usepackage[all]{hypcap}

% ++ use for refercence in the local arae \cref, e.g. \cref{fig:xyz}
%  has to come after hyperref package
\usepackage[noabbrev,capitalise,nameinlink]{cleveref}

% ++ use multiple figures/tables in one
\usepackage{caption}
\usepackage{subcaption}
\captionsetup[table]{position=top}
\captionsetup[figure]{position=bottom}
\captionsetup[subtable]{position=bottom}
% will result in references (typeset with \ref )
% like ‘1a’ but sub-references (typeset with \subref) like ‘(a)’.
\captionsetup{subrefformat=parens}


% ++ Enables glossaries-extra. Should be used for abbreviations in the paper.
\usepackage[abbreviations,shortcuts=true]{glossaries-extra}
% abbreviationstyle for acronyms
\setabbreviationstyle{long-short}
% one can also use \makeglossaries and \printglossaries however then you need to
% create a latexmk file and it might become more complicated for Windows users...

% put in this file your abbreviations
% add your abbreviations here
\newabbreviation{hbp}{HBP}{Human Brain Project}
\newabbreviation{mds}{MDS}{Multidimensional Scaling}
\newabbreviation{nlp}{NLP}{Natural Language Processing}
\newabbreviation{pos}{POS}{Part-of-speech}
\newabbreviation{rl}{RL}{Reinforcement Learning}
\newabbreviation{sr}{SR}{Successor Representation}
%%%%%%%%%%%%%%%%%%%%%%%%%%%%

% supress messages because of underful hbox, this is not a problem
\hbadness=10000



% some useful commands
\makeatletter % let's define a single dot
\DeclareRobustCommand\onedot{\futurelet\@let@token\@onedot}
\newcommand{\@onedot}{\ifx\@let@token.\else.\null\fi\xspace}
\makeatother

\newcommand{\etal}[1]{#1~et~al\onedot}
\def\accto{acc.~to\xspace}
\newcommand{\eg}{e.\,g.,\xspace}
\newcommand{\Eg}{E.\,g.,\xspace}
\newcommand{\cf}{cf\onedot}
\newcommand{\ie}{i.\,e.,\xspace}
\newcommand{\wrt}{w.\,r.\,t\onedot}
\newcommand{\aka}{a.\,k.\,a\onedot}
%\newcommand{\todo}[1]{\textcolor{red}{TODO: #1}}
\renewcommand{\vec}[1]{\bm{#1}}



% Sets the bib file

\addbibresource{literature.bib}

% When writing a large document, it is sometimes useful to work on selected sections of the document.
% Use this command to only build the document partially. Speeds up the developement cycle.
% For the final product, this has to be commented out.
%\includeonly{introduction,appendix,foo,bar}


\pagenumbering{roman}

\begin{document}
\clearpage
% %CONFIG
% This is for students' theses
  \begin{deckblatt}
    \Titel{The hippocampus and language: Word to word prediction in terms of the successor representation} % Title
    \Name{Rost} % Last name
    \Vorname{Philipp} % Given name
    \Geburtsort{Fürth} % Place of birth
    \Geburtsdatum{24. May 1996} % Date of birth
    \Betreuer{Paul Stöwer M. Sc., Prof. Dr.-Ing. habil. Andreas Maier, Dr. rer. nat. Patrick Krauß (Neuroscience Lab, University Hospital Erlangen)} % Advisor
    \Start{1st June 2022} % Start of thesis
    \Ende{\today} % End of thesis
    %\ZweitInstitut{ZweitInstitut} % Cooperation partner
\end{deckblatt}


\cleardoublepage


Ich versichere, dass ich die Arbeit ohne fremde Hilfe und ohne Benutzung
anderer als der angegebenen Quellen angefertigt habe und dass die Arbeit
in gleicher oder "ahnlicher Form noch keiner anderen Pr"ufungsbeh"orde
vorgelegen hat und von dieser als Teil einer Pr"ufungsleistung
angenommen wurde. Alle Ausf"uhrungen, die w"ortlich oder sinngem"a"s
"ubernommen wurden, sind als solche gekennzeichnet.
\\

Die Richtlinien des Lehrstuhls f"ur Studien- und Diplomarbeiten
habe ich gelesen und anerkannt, insbesondere die Regelung des
Nutzungsrechts. \\[15mm]
Erlangen, den {\selectlanguage{german} \today} \hspace{6.0cm} \\[10mm]



\cleardoublepage


\begin{center}
\bfseries
% Abstract in German
{\selectlanguage{german}"Ubersicht}
\normalfont
Den theoretischen Hintergrund der Masterarbeit bilden die Ort- und Gitterzellen des Hippocampus, die für verschiedenste Aufgaben der Orientierung zuständig sind. Das reicht von abstrakten Zuordnungen wie der Höchstgeschwindigkeit zu einem Fahrzeug auf Grundlage der Motorleistung und des Gewichts bis zur klassischen räumlichen Navigation in einer Stadt oder einem Gebäude. Da diese Resultate bereits per Maschinellem Lernen untersucht wurden, soll diese Arbeit davon handeln, ob diese Methoden auch dazu verwendet werden können, um Sprache zu verarbeiten, damit so ggfl. Rückschlüsse auf die Orts- und Gitterzellen gezogen werden können. Zu diesem Zweck soll die Theorie der Projektiven Karten und deren mathematischer Formulierung der Successor Representation genutzt werden. Um dies zu erreichen, werden mehrere Architekturen eines Neuronalen Netzes untersucht und verschiedene Techniken des Natural Language Processing verwendet, wobei das Hauptaugenmerk auf der Verarbeitung von Büchern liegt, mit denen die Trainingsdaten generiert werden können, da sie plausible Sprachdaten darstellen.
\end{center}


\vspace{5.0cm}

\begin{center}
\bfseries
% Abstract in English
{\selectlanguage{english}Abstract}
\normalfont
The theoretical background of my master thesis is founded on the concept of the place and grid cells of the hippocampus, which control different tasks of orientation. They range from classical spatial navigation in a city or building to abstract mappings between velocity and vehicle based on engine power and mass. These results were already examined by the means of Machine Learning. Thus, this work wants to try to extent the methods to process language in hope to gain some conclusions on place and grid cells. For this purpose, the Successor Representation as application of the Projective Map theory is implemented by deploying a Neural Network under multiple architectures. Furthermore, different techniques of Natural Language Processing are used, because the training data is generated from two books to have plausible language data.
\end{center}


\cleardoublepage

\tableofcontents

\cleardoublepage \pagenumbering{arabic}

\chapter{Introduction} \label{ch: introduction}
Understanding and therefore modeling the human brain is a challenge as old as science itself. Through the centuries, mankind contemplated and associated the brain as a complex version of the technology they were surrounded by, starting by comparing it with an abacus up to the \gls{hbp}\footnote{https://www.humanbrainproject.eu/en/} founded by the EU, which tries to simulate and build a ``silicon brain''. The first steps toward this venture were taken by Santiago Ramón y Cajal, who was rewarded with the Nobel Prize in $ 1906 $~\cite{Nobelprize06}. Furthermore, his research includes a full description of a nerve cell and further related concepts, for instance that signals are processed mono-directionally. These results lead directly to Rosenblatt's perceptron~\cite{Rosenblatt58P} and from then on to the nowadays modern field of Deep Learning in Computer Science.

Next to large-scale research projects like the \gls{hbp}, there also exist smaller ones tackling different parts of the brain, for example the hippocampus (\secreff{sec: Hippocampus}). It plays a fundamental role in forming new memories, processing emotions as part of the limbic system and in positioning, a result awarded with the Nobel Prize in $ 2014 $ (\secreff{sec: Hippocampus} \& \secreff{sec: predictive map theory})~\cite{Nobelprize14}. This abstract space, in which the navigation happens, is called \cognitiveroom{} (\secreff{sec: Hippocampus}) and \etal{Stachenfeld} provide an adequate mathematical theory, the \gls{sr} (\secreff{sec: SR}), to transfer the concept into a framework to work and do experiments with.

This master thesis tries to extend the application of the \gls{sr}, which is focused on spatial coordination in~\cite{StBoGe17HPM} to learning languages. For this purpose, some techniques of \gls{nlp} are useful, necessary and applied (\secreff{sec: data preparation}). One motivating indication this plan might work out is given by \etal{Stachenfeld}, who showed that their theory doesn't need well structured surroundings like a city but a topological/graphical environment is sufficient to retrieve viable results. Therefore, P. Stöwer did promising first experiments~\cite{Stöwer21MA}, which will be used as a foundation to generalize his framework (\chapreff{ch: framework}) in an attempt to gain more precise results (\chapreff{ch: results}).
   % Introduction
\cleardoublepage
\chapter{Theoretical Background}

% ======================================

\section{Hippocampus} \label{sec: Hippocampus}
The hippocampus is located in the brain and part of an old area called the archicortex. It is named after the greek word for seahorse, because it has the shape of one (\figref{\ref{wrapfig: Hippocampus and seahorse}}). This brain area can be divided into three parts: the dentate gyrus, the cornu ammonis and the subiculum~\cite{ORFrHa20CCN, GarzorzStark18BN}.

\begin{wrapfigure}{r}{0.35\textwidth}
	\centering
		\includegraphics[width=0.35\textwidth]{Hippocampus_and_seahorse.jpg}
	\caption{Hippocampus and seahorse~\cite{Seress10H}} % TODO Besser formatieren oder entfernen
	\label{wrapfig: Hippocampus and seahorse}
\end{wrapfigure}
The hippocampus plays a fundamental role in forming new memories (not preserving them, which is done across the brain) and is highly capable of learning new information fast. Regarding its functions, one was already mentioned: It is the key area when it comes to establishing new memories. Patients with a damaged hippocampus, therefore lacking this ability, will lose spatial and temporal orientation. Moreover epilepsy, schizophrenia and Alzheimer's disease are connected to this dysfunctional organ~\cite{Trepel17N}.
The hippocampus is also important in emotional contexts because it is an integral unit of the limbic system~\cite{GarzorzStark18BN}. Another task, and for this thesis the most important one, is navigation/orientation, not just in spatial surroundings, but also in an abstract context, called \cognitiveroom{}. Some examples for abstract contexts are: Danger of animals based on their appearance and speed of vehicles based on their weight and engine (\secreff{sec: predictive map theory}). To achieve this skill, two types of cells in the hippocampus are active: place cells and grid cells.
The first one encodes states/positions (one for each cell) and the latter resembles a coordinate system.
\paragraph{Place cells} \label{par: Place cell}
Place cells are irregular distributed across the cognitive room. Their firing is tied to the location of the state, whereby the term location has not always its classic spatial meaning if we navigate in an abstract setting (as mentioned above). The place cell is active in case we encounter the associate state. As seen in \figref{\ref{fig: Rat in maze}}, different place cells (each is color coded) fire at different positions in the parkour \eg turquoise is undoubtedly related to the first arch, meaning its activity spikes while the rat passes by. The remark of the thesis lies on place cells.
\begin{figure}
	\centering
		\includegraphics{Ortszelle_Beispiel.png}
	\caption{Activity pattern of color encoded place cell across a maze. Each place cell is exactly related to one distinct position of the corresponding environment \eg turquoise to the first arch. Its activity spikes if the rat walks along the arch~\cite{Stuartlayton13}.}
	\label{fig: Rat in maze}
\end{figure}
%This means in a spatial scenario, where we walk around in a square \TODO[fancyline]{Bild in GeoGebra erstellen mit Quadrat, Brunnen, Baum..., daneben Aktvierungsmuster der Ortszellen} (i.e. with a fountain), it fires irregularly and always then if we are at the position the place cell encodes. In case it resembles the fountain it will fire if we are close to it. \\
\paragraph{Grid cells}
This type of cell can be found in the entorhinal region and satisfies a more general purpose. They are regularly distributed and form a triangular lattice (\figref{\ref{fig: Grid cells}}). It provides raw spatial information in terms of a metric or distance measure the hippocampus integrates with the place cells~\cite{ORFrHa20CCN, BellmundEtAl18NC}.
\begin{figure}
	\centering
		\includegraphics[scale=0.25]{Gitterzelle_Beispiel.png}
	\caption{Sketched path of a rat moving in a square, while tracking firing grid cells. As their name suggests, they form a regular lattice over the space. Hence, they act as coordinate system. The information provided by grid cells is combined with that of the place cells to generate a full picture of the surroundings~\cite{Moser15PGM}.}
	\label{fig: Grid cells}
\end{figure}


% ======================================

\section{Predictive map theory} \label{sec: predictive map theory}
To explain the principle of the predictive map theory introduced in~\cite{StBoGe17HPM}, it is necessary to illustrate the concept of a \cognitiveroom{}, mentioned before in \secreff{sec: Hippocampus}. An example is of course a naive navigational task as presented by \etal{Stachenfeld} and similar to the setting of \figref{\ref{fig: Rat in maze}}. The authors even demonstrated that just a topological environment is sufficient to craft a \cognitiveroom{} and apply the predictive map theory.

The concept becomes far more interesting when talking about cognitive rooms founded on experience \ie the speed of vehicles based on weight and engine specifications. This category of a \cognitiveroom{} also fits the topic of the thesis much better, since it aims to model language not a spatial environment. An illustrating example can be found in \figref{\ref{fig: vehicles cognitive room}}. For instance, a ``sports car'' might be rather lightweight but has plenty of horse power.
\begin{figure}
	\centering
		\includegraphics[width=0.4\textwidth]{Fahrzeuge_Cognitive_Room.jpeg}
	\caption{Exemplary \cognitiveroom{} of vehicles according to their weight and engine power. An unknown car can be placed easily in the environment given the two parameters because there are already established place cells acting as abstract waypoints (the depicted cars) to support the orientation \ie finding its place on the map. By doing so, it is immediately possible to derive information about the appearance of the automobile~\cite{BellmundEtAl18NC}.}
	\label{fig: vehicles cognitive room}
\end{figure}
By using these two characteristics, the \cognitiveroom{} has the shape of a $ 2d $-plane. For instance, while reading about an alien car, it is immediately possible to compare it with different well-known vehicles and draw conclusions about its shape since the \cognitiveroom{} has enough information to position the car within it. All these decisions of placing new objects in an appropriate context is done by place and grid cells (\secreff{sec: Hippocampus}). Expanding the example by the firing of cells results in the full illustration given in \figref{\ref{fig: vehicles with place and grid cells}}.
\begin{figure}
	\centering
		\includegraphics[width=1.\textwidth]{Fahrzeuge_Ortszelle_Gitterzelle.jpeg}
	\caption{Left: \cognitiveroom{} of vehicles according to weight and horse power. Middle: Firing pattern of place cells crafting the \cognitiveroom{} \ie the boundaries in \figref{\ref{fig: vehicles cognitive room}}. Right: Corresponding lattice of grid cells.~\cite{BellmundEtAl18NC}}
	\label{fig: vehicles with place and grid cells}
\end{figure}

%
% Paul bezieht die „Prdictive Map theory“ lediglich „auf die Autos und Tiere“
%

%In case of encountering a unknown species the predictive map becomes active in terms of firing place cells encoding animals sharing the same features and thus evaluating the potential danger. In these terms a place cell doesn't encode the current state but a future state because the new living being will be cataloged next to the known ones. So, the cognitive room exhibits a predictive property.
%
%We can, for instance, apply it to the scenario of the rat passing a maze as seen in Fig. \ref{fig: Rat in maze}. In par. \nameref{par: Place cell} was mentioned that a place cell resembles the current position. By the new perspective follows that the cell fires beforehand~\eg again in terms of the turquoise cell: This cell is now contemplated active if the rat is within the are preceding the arch and aims to enter it.
%
%Furthermore, the authors argue that


% ======================================

\section{Successor Representation} \label{sec: SR}
According to \etal{Stachenfeld}, our behavior in an open spatial environment, or in general in a \cognitiveroom{} \eg a city, follows the predictive map theory introduced in \secreff{sec: predictive map theory}. An active place cell encodes the next/successor state entered by the agent.

To model this or the general setting of predicting future states, the \gls{sr} was developed by a \gls{rl} approach. Furthermore, the \gls{sr} and the predictive map theory go hand in hand. The latter is an application regarding the former: The authors support the proposition that hippocampal mechanics, explained by the predictive map theory, can be described via the \gls{sr}.

% ======================================

\subsection{Mathematical Foundation}
The basis lies largely in \gls{rl}, in formula:
\begin{equation}\label{eq: rl}
	V(s) := E \left[
				\sum_{t=0}^{\infty}
					\gamma^t R(s_t) | s_0 = s
			\right]
\end{equation}
with $ V $ resembling a value function, expressed via the reward function $ R $, which operates on state $ s_t $, encoded by the sum over $ t $, starting in $ s $. $ \gamma \in [0,1] $ serves as a discount factor to control the influence of states reached in distant future. High values permit distal states to play a larger role, whereas smaller values de facto limit the result to neighboring positions\footnote{Short mathematical explanation: $ p^t \xrightarrow[]{t \to \infty} 0 $ for $ p \in [0,1) $, the greater $ p $ the slower happens the approach of the limit. For $ p = 1 $ the sequence is constant. In our case every state is taken into account equally.}. By the reward function is obtained how beneficial the currently visited state $ s_t $ is. % The expected value is taken because there are \wlog plenty of paths starting in $ s $.
After the calculation of $ V $, the function can be decomposed into a more intuitive representation, consisting of a state matrix $ M $, called the \srmat{}, and the known reward function $ R $:
\begin{equation}\label{eq: v with sr-matrix}
	V(s) = \sum_{s'}
				M(s, s') \cdot R(s')
	\text{.}
\end{equation}
The first argument of $ M $ specifies the row, the latter the column. Each cell contains the discounted expected number of times the agent visits state $ s' $ starting from $ s $. Additionally, \etal{Stachenfeld} mention that the \srmat{} can be derived from a transition probability matrix $ T $ for the positions $ s $~\cite{StBoGe17HPM}. Having $ T $, it follows
\begin{equation}\label{eq: sr or m via T}
	M = \sum_{t = 0}^{\infty}
			\gamma^t T^t
	\text{,}
\end{equation}
which is a geometric series and converges for $ \gamma < 1 $ towards
\begin{equation}\label{eq: geo series}
	(I_n - \gamma T)^{-1}
	\text{,}
\end{equation}
where $ I_n $ is the corresponding identity matrix.

Although defining all formulae by infinite sums, it is seamlessly possible to calculate the \srmat{} in \equref{\eqref{eq: sr or m via T}} up to a finite index or starting at an arbitrary $ t $ \ie $ t=1 $. Doing so makes sense in a language environment. The identity matrix would imply that a word can follow itself, which is extremely rare\footnote{Although sentences like ``Ich hoffe, dass das das Richtige ist.'' do occure in german.}. Therefore, the first summand will always be $ \gamma T $, where $ T $ is calculated by a Neural Network (\secreff{ch: framework}). If the indices in \equref{\eqref{eq: sr or m via T}} are altered, the limit of the geometric series no longer applies directly and has to be adjusted by subtracting the first summands from \equref{\eqref{eq: geo series}}. The \gls{sr} and thus the depicted formulas, especially the \srmat{} and the transition probability matrix, are policy dependent. This is reflected by the training data.

By definition, matrix $ M $ reveals all successor states with their particular probability because the summation combines the following positions, which are calculated by exponentiation, into one matrix. By examining a row (\figref{\ref{fig: sr-spalte}}) \eg row $ k $, it is possible to follow all paths starting from state $ k $.

One advantage of the \gls{sr}, \ie describing the model by the \srmat{} $ M $, is its high flexibility regarding the evaluation of different reward functions given by \equref{\eqref{eq: v with sr-matrix}}. The value of a state $ s $ can be calculated in an instant with a different reward function while no relearning is necessary.

\paragraph{\gls{sr} and grid cells}
Although not further discussed in the thesis but an interesting claim of \etal{Stachenfeld} is that the eigenvalue decomposition of the \srmat{} reveals the grid cell structure. They provide supplementary information depicting many examples~\cite{StBoGe17HPM}.

% ======================================

\subsection{Example for the Successor Representation}
This subsection is dedicated to fill the concept of the \gls{sr} and the \srmat{} $ M $ with some intuition. \etal{Stachenfeld} simulated a linear spatial environment built by six states with a simple policy merely consisting of two actions the agent can apply: Going one step to the right or pausing.
%
\paragraph{Rows}
In this scenario, a plot of single rows of $ M $ is shown in \figref{\ref{fig: sr-zeile}}.
\begin{figure}
	\centering
		\includegraphics[width=0.7\linewidth]{Beispiel_SR_Zeile.png}
	\caption{Schematic plot of the rows of state $ s^1 $ and $ s^5 $ respectively. By interpreting the ordinate values as probabilities for transitioning instead of a probability for the current state, it is possible to make assumptions on the future path the agent may take. Hence, matrix $ M $ describes all possible paths. In both cases the policy prefers pausing over changing the state.}
	\label{fig: sr-zeile}
\end{figure}
From the upper half of \figref{\ref{fig: sr-zeile}}, examining the row of $ s^1 $, it is possible to deduce that the agent will most likely remain at its current position, with the values for distal locations disappearing. A different point of view results for the part of $ M(s^5, s^i) $. It is obvious that going backwards is nothing to reckon with since the numbers for transitioning distribute over $ s^5 $ and \texttt{goal} by slightly favoring the former. This behavior was expected by the policy.
%
\paragraph{Columns}
It is also worth analyzing the columns of $ M $, called \emph{place fields} by \etal{Stachenfeld} (\figref{\ref{fig: sr-spalte}}). Having the policy in mind, it is no surprise that the values $ M(s^i, s^5) $ ascend in parallel to the index $ i $. The probability for entering $ s^5 $ grows by approaching it. In addition the plot shows how $ s^5 $ is probably reached best, simply by passing via $ s^3 $ and $ s^4 $. This might seem obvious, but in a more complex \cognitiveroom{} the graph won't look as ordinary and therefore will contain plenty of distributed information. %A spike on $ s^1 $ would counteract it because it means stepping from $ s^1 $ to $ s^5 $ directly (as seen in Fig. \ref{fig: sr-zeile} this is not the case).
\begin{figure}
	\centering
		\includegraphics[width=0.7\textwidth]{Beispiel_SR_Spalte.png}
	\caption{Schematic plot of the $ s^5 $-column depicting how $ s^5 $ is reached by ascending probabilities. It is possible to recapitulate the policy consisting of pausing or taking one step to the right. Entering $ s^5 $ is most likely from $ s^4 $ and $ s^5 $ (due to resting). }
	\label{fig: sr-spalte}
\end{figure}


% ======================================

\section{Multidimensional Scaling}
The goal of \gls{mds} is to calculate a $ m $-dimensional mapping, $ m < n $, of a given point cloud in $ \mathbb{R}^n $ that preserves the original distances as good as possible~\cite{HaTiFr17ESL}. The result of the calculation is unique modulo rotation and scaling. Therefore, it is based on a metric and not exact coordinates. \gls{mds} is used to analyze similarities between the rows of the \srmat{} by determining clusters in the graph.\\
%
The algorithm is simple and only uses basic linear algebra. \gls{mds} works with a distance matrix $ D $, where each entry is equal to $ d_{ij}^2 $, the squared distance between two points $ x_i, \ x_j \in \mathbb{R}^n $, whose coordinates are (in principle) unknown. By double centering $ D $ it is possible to calculate the matrix product $ X^\top X $, where $ X $ bears the coordinates in the desired dimension~\cite{Riess20PA}. Double centering means multiplying by a matrix $ C := I_n - \frac{1}{n}J_n$, where $ J_n $ is a $ n \times n $-matrix of ones:
\begin{equation}
	\underbrace{-\frac{1}{2} CDC}_{B :=} = X^\top X
	\text{,}
\end{equation}
The centering matrix $ C $ has, after a multiplication with a column vector, the same effect of subtracting the mean of all components from the vector itself. \\
In the next step, the $ m $ largest eigenvalues of $ B $ are calculated along with their corresponding eigenvectors. Finally, the $ m $-dimensional coordinates are determined:
\begin{equation}
	X_m = E_m V_m^{1/2}
	\text,
\end{equation}
where $ E_m $ contains the $ m $ eigenvectors and $ V_m $ is a $ m $-dimensional diagonal matrix with the associated eigenvalues.


% ======================================

\section{Metric for quantifying the results} \label{sec: metric}
Throughout the presentation of the results in \chapreff{ch: results} numerous matrices and \gls{mds} plots are used. Sometimes, they give a clarifying visual response, but not in all cases. When comparing different approaches, images lack the needed objectivity and plausible criteria to rate the outcomes. To tackle this issue, a metric was developed to have the possibility to draw objective conclusions.

Since Neural Networks on languages are trained, a measure on the grade of the closeness to the real counterpart is necessary, which is referred by ``ground truth (distribution)'' in the following. The mathematical objects are in both cases squared matrices of dimension $ n \in \mathbb{N} $ built by transposed probability vectors. Nevertheless, the presented mapping is made for $ n \times m $-matrices. Depending on the model type, the ground truth vectors are \onehot{s} or share different fractions across all entries (\secreff{sec: w2w models}).

Hence, the starting positions for the metric are probability vectors. The obvious way to quantify the results is by taking the euclidean norm $ d $ of the difference of the ground truth and the prediction. Consequentially, $ d $ takes values between $ 0 $ and $ \sqrt{2 \cdot n} $ because the maximal difference for each row is $ \sqrt{2} $ and there are $ n $ in total. $ \sqrt{2} $ is derived by the following nonlinear program
\begin{align}
	\mathrm{max} 	& \quad \Vert \vec{x} - \vec{y}\Vert_2 = \sqrt{\sum_{i=1}^{n} (x_i - y_i)^2} \nonumber\\
	\mathrm{s.t.} 	& \quad \sum_{i=1}^{n} x_i = 1, \ \sum_{i=1}^{n} y_i = 1\\
					& \quad x_i, y_i \in [0,1] \nonumber
	\text{,}
\end{align}
which is solved by \onehot{s} for $ \vec{x} $ and $ \vec{y} $ where $ x_i = 1 \neq y_i$ for a $ i \in \{1, \ ..., \ n\} $. Or to put it bluntly, the difference takes its highest values for all scenarios in which the vectors $ \vec{x} $ and $ \vec{y} $ are perpendicular and have a maximal euclidean norm, which means being a \onehot{}. This relation is present $ n $-times for the ground truth matrix and the learned one, implying that the maximal difference is $ \sqrt{2 \cdot n} $.

Finally, it is possible to define the metric on the set $ \mathcal{P} $ of $ n \times m $-probability matrices:
\begin{equation}
	d_A \colon \mathcal{P} \to [0,1], \qquad L \mapsto \frac{1}{\sqrt{2 \cdot n}}\Vert A - L \Vert_2
	\text{,}
\end{equation}
where $ A $ describes a fixed matrix in $ \mathcal{P} $. In the scope of this work the ground truth will play the role of $ A $. By $ d_A $, the learned matrix $ L $ is mapped to $ 0 $ if it matches the ground truth distribution perfectly and to $ 1 $ if the rows satisfy the conditions mentioned above.

% ======================================

%\subsection{Root-Mean-Square-Error}
%%\section{\gls{rmse}}
%The metric presented in the section beforehand is a generalization of the \gls{rmse}, which is also used for evaluation, especially in \secreff{sec: average approach}. It is defined as
%\begin{equation}
%	\mathrm{RSME} \colon \mathbb{R}^n \to \mathbb{R}_{\ge 0}, \qquad \vec{x} \mapsto  \sqrt{\sum_{i=0}^{n} \frac{x_i^2}{2}}
%	\text{.}
%\end{equation}
%The results of the models in \secreff{sec: average approach} are clearer to observe, so it is possible to analyze them more precisely \ie row-wise. Because the the objects of interest are also transition probability matrices, the connection between the lowest and highest values are preserved, if the difference between ground truth and learned row vector is plugged in. Hence, the range of the \gls{rmse} will be $ [0, 1] $. The equivalence between $ d $ and $ \mathrm{\gls{rmse}} $ follows for $ n = 1 $.


% ======================================

\cleardoublepage
\chapter{Framework} \label{ch: framework}
After setting up the theoretical basis for the tools needed to start the experiments, the code framework will be introduced. In general, it is a supervised dense Neural Network written in \python{}~\cite{VanRossumEtAl09Python} with the help of \keras{}~\cite{chollet2015keras} and \numpy{}~\cite{harris2020array}. The process is divided into two phases: One training phase and one for visualization of the results. The goal was being capable to calculate a decent \srmat{} showing visible clusters in the \gls{mds}-plot. To retrieve the \gls{sr}, a Neural Network is trained, whose predictions serve as transition probability matrix $ T $. If not otherwise stated, a discount factor $ \gamma = 0.5 $ is used.

Different scenarios were tested, hence various models were configured having distinct features \eg some work with \onehot{s}, while other use word vectors or made up rules and datasets.

% ======================================

\section{First Model and Architecture} \label{subsec: first model and architecture}
The first class of networks augments the results in~\cite{Stöwer21MA}. P. Stöwer's models rely on predefined rules, like

\begin{itemize}
\phantomsection
\label{enum: rule set}
	\item \texttt{Adjective → Noun}
	\item \texttt{Verb → Adjective}
	\item \texttt{Personal Pronoun → Verb}
	\item \texttt{Question word → Personal Pronoun}
\end{itemize}
for building the dataset backed by a word database containing the corresponding information. Starting point is the \cognitiveroom{}, which consists of a list reflecting the whole data. The training data was crafted in accordance by randomly choosing respectively one of the four rules above and within the word class by chance an example. This information is used to initialize a \onehot{} and is done for input and output of the network.

The goal was to attain results on the behavior of the model if it is extended by more rules and words. One can imagine this type of model as a graph (\figref{\ref{fig: first model graph}}).

\begin{figure}
    \centering
    \includegraphics[scale=0.35]{Bilder/Graphen/first_model_graph2.png}
    \caption{The first two rules depicted as graph. In gray are corresponding \onehot{s} for the exemplary cognitive room \texttt{[blue, to run, desk]} denoted. The rules serve as edges and the word classes as vertices.}
    \label{fig: first model graph}
\end{figure}


% ======================================

\section{Word to word models} \label{sec: w2w models}
The more difficult challenge lies in unannotated texts without a paradigm for pairing words of an example data set. In this manner a language normally occurs and is learned by humans. A priori one has to expect results of poorer quality in comparison to the configuration of \secreff{subsec: first model and architecture} because they were tailored and \gls{nlp} comes always with uncertainties.

% - - - - - - - - - - - - - - - - - - -

\subsection{Data preparation} \label{sec: data preparation}
The data is extracted from two books, namely ``Gut gegen Nordwind'', written by Daniel Glattauer in German~\cite{Glattauer06GGW} and from Jostein Gaarder ``Sophie's World''~\cite{Gaarder96SW} in English. Two languages were chosen since German comes in general with a high degree of freedom in word order, in comparison English is more restrictive. This distinction may be important, since analyzing successive words is fundamental for this work. Because the books are available as \texttt{pdf}-file, the python module \pymupdf{}~\cite{pymupdf} is used to generate a simple \texttt{String} containing the whole text, which is afterwards parsed by \spacy{}~\cite{spacy2}. This is a powerful tool in the area of \gls{nlp} and some techniques are indispensable for further analysis, mainly
\begin{itemize}
	\item Tokenization: segmenting text into words, punctuations marks etc.
	\item \gls{pos}-Tagging\footnote{More information on \cite{udpostags}}: assigning word types to tokens, like verb or noun \label{item: pos tag}
	\item Lemmatization: assigning the base forms of words\footnote{For example, the lemma of “was” is “be”, and the lemma of “rats” is “rat”.}
	\item word2vec~\cite{MikolovEtAl13DRW, MikolovEtAl13EEW}: calculating a vector representation with real values of a word, in the following called \emph{word vector}.
\end{itemize}
%Lemmatization and Part-of-speech (POS) Tagging are used for bookkeeping and result evaluation.
%
Additionally, a mechanism was implemented to extract an exact number of words having equal sized foundations in both languages. The training data consists of word pairs in their occurring order, for instance the sentence
\begin{quote}
	Goethe remarked about Alexander von Humboldt to friends that he had never met anyone so versatile.\footnote{Sentence taken from \cite{Wulf16ION}} %\footcite{Wulf16ION}
\end{quote}
gets tokenized, lemmatized and coupled having
\begin{verbatim}
	("Goethe", "remark"), ("remark", "about"), ("about", "Alexander"), ...
\end{verbatim}
where the first component serves as input and the second one as supervised output.

Clearly, not the actual word is fed into the Neural Network, but numerical representations: either a \onehot{} or a word vector. To construct the former, the concept of the \cognitiveroom{} is applied by building a list containing all words of the text \ie one word resembles one state and states are encoded by place cells. To learn the transition probability matrix, as proclaimed at the beginning of the chapter, one has to transform the prediction of the Neural Network into a probability vector via division by its sum. This processing is done not during training because it is supervised via \onehot{s}.

% --------------------------------------

\subsection{\onehot{} approach} \label{subsubsec: onehot approach}
This configuration follows the principles of the first model (\secreff{subsec: first model and architecture}) by using \onehot{s} as input and output but there are no invented grammatical rules anymore. The training data is now directly related to concrete words and not to a word class. An illustration of the data structure is given in \figref{\ref{fig: text model graph ohe w2v}}.

% --------------------------------------

\subsection{Word vector approach} \label{subsubsec: word vector approach}
The Neural Network takes word vectors, a $ 300d $-vector of real numbers, as input and omits them during training. Word vectors are calculated by \spacy{}. To be precise, this step has two stages. Building the training data is easy because it is effortlessly possible to retrieve the real valued vector given a word. Since predictions aren't (and can't be) as accurate as the results \spacy{} computes, it is impossible to query a dictionary or database to reshape the exact word. For such situations, the module offers the option to retrieve a list with the $ n \in \mathbb{N} $ closest words. This list includes the $ n $ words whose vector representations have the smallest euclidean distance to the desired vector \ie the prediction.

In the next step, a check is performed whether the word is part of the book \ie the \cognitiveroom{}. If so, the euclidean distance is taken as entry in the Transition Probability Matrix $ T $, which will undergo a row-wise transformation to fit the criteria of a transition probability matrix. One disadvantage is an ambiguous prediction and therefore a less sharp $ T $. But in comparison to the \onehot{}, a word vector bears a lot more information, which hopefully can be exploited by the neural network. The data structure is shown in \figref{\ref{fig: text model graph ohe w2v}} next to the \onehot{} equivalent.

\begin{figure}
    \centering
        \includegraphics[width=\linewidth]{Bilder/Graphen/w2w_ohe_w2v.png}
    \caption{The word to word models can also be illustrated as graphs, though more complex. The lemmatized words of the text serve as vertices and the pairs mentioned before in \secreff{sec: data preparation} as edges. In green are word vectors and in blue \onehot{s} denoted. This cropped graph is generated by the bold passages of the text: ``\textbf{Alice sends Bob} a message. \textbf{Alice goes} to the grocery store. Peter \textbf{sent him} a letter. \textbf{Bob went} to his friend.''}
    \label{fig: text model graph ohe w2v}
\end{figure}

% --------------------------------------

%\subsection{Combined approach} \label{subsubsec: combined approach}
%The third model blends in both approaches by using the word vectors as input and a \onehot{} as output. It's goal is to tackle the imprecision aspects of the inevitable blur the word vector configuration experiences.


% ======================================

\section{Average approach} \label{subsubsec: average approach}
Because the models become enormous and  especially thus the evaluation difficult, this configuration aims to analyze the results on a rougher scale by taking averages of the predictions. While collecting the training data, the \postag{} of each one is saved and after training the cumulative outputs of a word class prediction are taken \ie all \onehot{s} of a \texttt{VERB} are mapped by the Neural Network, averaged into one vector and then checked for the most probable \postag{s} (\secreff{sec: average approach}).

\postag{s} were mentioned before in \secreff{sec: data preparation} and are just another term for word classes. In detail, a subset of the UniversalDependencies \postag{s}~\cite{udpostags} are used (\tabref{\ref{tab: ud pos tags}}) because this project provides a rich dataset, good documentation and its guidelines are implemented by \spacy{}.
\begin{table}
	\centering
	\caption[Listing of \postag{s}.]{List of relevant \postag{s} including examples and short definition.}
	\begin{tabular}{cl|cl}
		\toprule
		\postag{} 	  & Definition \& Example 		& \postag{} 		& Definition \& Example \\
		\midrule
		\texttt{ADJ}  & Adjective: educated, hot 	& \texttt{VERB} 	& Verb: to run, to drink \\
		\texttt{ADV}  & Adverb: easily, everywhere	& \texttt{DET} 		& Determiner: this, a, no \\
		\texttt{NOUN} & Noun: car, bottle 			& \texttt{PART}		& Particle: 's, not \\
		\texttt{AUX}  & Auxiliary: to have, should 	& \texttt{ADP}		& Adposition (Pre- \& Postpositions): in, on \\
		\texttt{PRON} & Pronoun: she, ours 			& \texttt{REST}		& \parbox{7cm}{Rest (Container for Conjunctions and additional residuals): and, if}\\
		\bottomrule
	\end{tabular}
	\label{tab: ud pos tags}
\end{table}
%In the next step the $ n $ indices with the highest values are checked for their word class, so if index $ i $ is one of the $ n $ highest and encodes the word \texttt{fish}, its word class \texttt{Noun} is counted. Finally, one has constructed a vector for each word class where each component resembles the probabilities after dividing by $ n $ for the following word class. An estimation of the \gls{sr} could look like the matrix in \eqref{eq: example average sr matrix} for the three word classes \texttt{Verb}, \texttt{Adjective} and \texttt{Noun} \ie the predicted transition probability for \texttt{Verb → Noun} is equal to $ 0.8 $, etc.
%\begin{equation} \label{eq: example average sr matrix}
%	\begin{pmatrix}
%		0 & 0.2 & 0.8 \\
%		0.1 & 0.3 & 0.6 \\
%		0.7 & 0.2 & 0.1
%	\end{pmatrix}
%\end{equation}


\cleardoublepage
\section{Results}

% === Results – First model ===================================

\begin{frame}
\frametitle{Results – First model}
	\begin{itemize}
		\item<+-> Prediction works quite well i.e., the rules are recognizable e.g., {\huge \texttt{Adjective → Noun}}
		\item<2-> MDS plot shows clustered word classes 
	\end{itemize}
    \vspace*{-0.95cm}
	\begin{columns}
		% SPALTE 1
		\begin{column}{0.33\textwidth}
			\begin{figure}
			\centering
				\includegraphics[height=0.98\columnwidth]{Bilder/results_first_model/8Rules/plots/First Model + More Rules_100E_100BS_1L_1C/Transition_Probability_Matrix;_t=1,_DF=0.5.png}
			\end{figure}
            \begin{center}
                {\large Learned SR}
            \end{center}
		\end{column}
		% SPALTE 2
		\begin{column}{0.33\textwidth}
    			\begin{figure}
    				\centering
    					\includegraphics<2->[height=0.98\columnwidth]{Bilder/results_first_model/8Rules/plots/First Model + More Rules_100E_100BS_1L_1C/MDS_of_Transition_Probability_Matrix;_t=1,_DF=0.5.png}
    			\end{figure}
            \Os{2}{
                \begin{center}
                    {\large MDS plot}
                \end{center}
            }
		\end{column}
		% SPALTE 3
		\begin{column}{0.33\textwidth}
%			\vspace*{7mm}
			\begin{figure}
				\centering
					\includegraphics<3->[height=0.98\columnwidth]{Bilder/results_first_model/4Rules/plots/First Model + More Rules_100E_100BS_1L_1C/SR,_t=2,_DF=0.5.png}
			\end{figure}
        \Os{3}{
            \begin{center}
                {\large SR for $ t=2 $, less word classes}
            \end{center}
        }
		\end{column}
	\end{columns}
% --------------------------------------
\mynote{
\begin{itemize}
    \item[üü] Finally, I will present the results.
    \item[ü] Again, we'll begin talking about the First Model
	\item Works well because the SR performs best in these well defined scenarios
	\item[i] To avoid clutter only word classes are labeled, indeed one row corresponds to one word.
	\item The MDS plot shows clustered word classes, which also means learning was successful. Although not necessary for this type of model, it offers some visual feedback for configurations using a larger data set because their matrix can't be plotted
	\item LETZTES BILD: SR for t=2, it is possible to recognize the following states of a {\huge \texttt{question word}}, here they are {\huge \texttt{Personal Pronoun}} and {\huge \texttt{Verb}}
\end{itemize}
}
% SAINT MARTIN MEMMINGEN
\end{frame}

% === Results – Word to word models 1/2 ===================================

\begin{frame}
\frametitle{Results – Word to word models}
	\begin{itemize}
		\item<+-> Comparison with a ground truth/statistical assessment possible by a metric
	\end{itemize}
	\vspace*{-1.5cm}
	\begin{columns}
		% SPALTE 1
		\begin{column}{0.33\textwidth}
			\begin{figure}
				\centering
					\includegraphics[height=1.1\columnwidth]{Bilder/BspW2W/plots/OHE_OHE_500E_100BS_1L_1C_5P_30T_J/J_5pages_30T_words.png}
			\end{figure}
			\begin{center}
				{\large Ground truth, Transition probability matrix}
			\end{center}
		\end{column}
		% SPALTE 2
		\begin{column}{0.33\textwidth}
			\begin{figure}
				\centering
					\includegraphics[height=1.1\columnwidth]{Bilder/BspW2W/plots/OHE_OHE_500E_100BS_1L_1C_5P_30T_J/Transition_Probability_Matrix;_t=1,_DF=0.5.png}
			\end{figure}
			\begin{center}
                {\large Learned transition probability matrix}
			\end{center}
		\end{column}
		% SPALTE 3
		\begin{column}{0.33\textwidth}
            \vspace*{8.5mm}
			\begin{figure}
				\centering
					\includegraphics[height=1\columnwidth]{Bilder/BspW2W/plots/OHE_OHE_500E_100BS_1L_1C_5P_30T_J/MDS_of_Transition_Probability_Matrix;_t=1,_DF=0.5.png}
			\end{figure}
			\begin{center}
                {\large Learned MDS}
			\end{center}
		\end{column}
	\end{columns}
% --------------------------------------
\mynote{
\begin{itemize}
    \item[ü] word to word models show a different outcome.
    \item It is possible to compare the results to a ground truth/statistical assessment, hence the Metric $ d_A $ comes into play
	\item[i] On the left (LINKES BILD) is a ground truth depicted and next to it the predictions of the network. Although there is a resemblance visible it has to be assessed cautiously because a tiny data set was used for illustration purposes only to convey an intuition for the results and the procedure.
	\item[i] The MDS is displayed because matrices won't provide visual feedback anymore and as seen before sufficient learning is also visible in the cluster plot.
\end{itemize}
}
\end{frame}

% --- Results – Word to word modeles 2/2 -----------------------------------

\begin{frame}
	\frametitle{Results – Word to word models}
	\begin{itemize}
		\item<1-> It is possible to compare the results to a ground truth/statistical assessment $\Longrightarrow $ Metric $ d_A $
		\item<2-> Surprisingly \onehot{s} outperform word vectors i.e., word vectors are just bad
		\item<3-> German or english doesn't make that much of a difference
	\end{itemize}
	\begin{columns}
		% SPALTE 1
		\begin{column}{0.4\textwidth}
    \Os{1}{
			\vspace*{3mm}
			\begin{table}
%				\caption{Configurations with metric w.r.t. ground truth}
				\begin{tabular}{ll}
					\toprule
					Version					& Metric \\
					\midrule
					german, \onehot{} 		& $ 0.08 $	\\% 0.08200
					german, word vector		& $ 0.74 $	\\% 0.74139
					english, \onehot{}		& $ 0.10 $	\\% 0.10021
					english, word vector	& $ 0.78 $	\\% 0.77522
					\bottomrule
				\end{tabular}
			\end{table}
			\begin{center}
				{\large Configurations \& metric w.r.t. ground truth}
			\end{center}
            \vspace{3.2cm}\hspace*{5mm}\parbox{0.9\columnwidth}{{\normalsize \notsoimportant{If you want to know more about the metric, you can ask after the talk}}}
    }
		\end{column}
		% SPALTE 2
		\begin{column}{0.3\textwidth}
			\vspace*{-1cm}
			\begin{figure} % ohe de
				\centering
					\includegraphics<4->[height=0.55\textheight]{Bilder/W2W/OHE_OHE_5000E_100BS_1L_1C_200P_1500T_D/MDS_of_Transition_Probability_Matrix;_t=1,_DF=0.5.png}
			\end{figure}
    \Os{4}{
			\begin{center}
				{\large MDS of german, \onehot{}}
			\end{center}
    }
		\end{column}
		% SPALTE 3
		\begin{column}{0.3\textwidth}
			\vspace*{-1cm}
			\begin{figure} % w2v de
				\centering
					\includegraphics<4->[height=0.55\textheight]{Bilder/W2W/W2V_W2V_5000E_100BS_1L_1C_200P_1500T_D/MDS_of_Transition_Probability_Matrix;_t=1,_DF=0.5.png}
			\end{figure}
    \Os{4}{
			\begin{center}
				{\large MDS of german, word vectors}
			\end{center}
    }
		\end{column}
	\end{columns}
\mynote{
	\begin{itemize}
		\item In the table we find all four configurations comprising of \onehot{s} and word vector each ran with german and english.
        \begin{itemize}
            \item[i] The metric maps matrices close to $ 0 $ if the equal the ground truth and to $ 1 $ if not
        \end{itemize}
		\item Surprisingly \onehot{s} perform better than word vectors. Since they encode a word with more than two numbers, this was nothing to reckon with.
		\item We expected that english performs better than german due to the more static word order which was a fallacy
		\item In the image on the left is the MDS plot of the SR using \onehot{s} illustrated and on the right using word vectors. In both cases the hoped clusters didn't form and the configuration on the right seems to be omit a degenerated output at all. 
	\end{itemize}
}
\end{frame}

% === Results – Averaging models 1/3 ===================================

\begin{frame}{Results – Averaging models}
    \begin{itemize}
        \item<1-> Outcome of the plain vector models wasn't satisfying (as seen in the MDS plots), so averaging was established
    \item<2-> Results were indeed exploitable i.e., word class transition probabilities are partially reflected very accurately
    \end{itemize}
    \begin{columns}
        % SPALTE 1 ADP barplot
        \begin{column}{0.5\textwidth}
            \vspace*{-1cm}
            \begin{figure}
                \centering
                    \includegraphics<2->[width=0.66\textheight]{Bilder/Barplots/Avg_OHE_OHE_5000E_100BS_1L_1C_200P_1500T_D/Combined_Barplot_ADJ_F.png}
            \end{figure}
        \end{column}
        % SPALTE 2 ADJ barplot
        \begin{column}{0.5\textwidth}
            \vspace*{-1cm}
            \begin{figure}
                \centering
                    \includegraphics<2->[height=0.66\textheight]{Bilder/Barplots/Avg_OHE_OHE_5000E_100BS_1L_1C_200P_1500T_D/Combined_Barplot_ADP_S.png}
            \end{figure}
        \end{column}
    \end{columns}
% --------------------------------------
\mynote{
\begin{itemize}
    \item[ü] After all, coming to the Average Approach
    \item Because the outcome of the plain vector models wasn't satisfying (as illustarted in the MDS plot beforehand), an averaging approach was developed. Based on the predictions shown beforehand, transition probabilities between word classes were calculated
    \item The results are indeed exploitable especially using \onehot{s}. There, word class transitions are partially reflected very accurately as we can see in both bar plots
    \item[i] In both figures the learned probabilities and the ground truth probabilities are plotted, in blue and green respectively. The title is referring to the first word class, whereas the x axis shows the particular successor. Both diagrams show good accuracy for the word classes. 
\end{itemize}
}
\end{frame}

% --- Results – Averaging models 2/? -----------------------------------

\begin{frame}{Results – Averaging models}
    \begin{itemize}
        \item Matrices are $ 10 \times 10 $, so we display them
    \end{itemize}
    \vspace*{-1.5cm}
    \begin{columns}
        % SPALTE 1
        \begin{column}{0.33\textwidth}
            \begin{figure}
                \centering
                \includegraphics[width=1.1\columnwidth]{Bilder/average_models/ground_truths/D_200pages_1500T_tags.png}
            \end{figure}
            \begin{center}
                {\large ground truth (german)}
            \end{center}
        \end{column}
        % SPALTE 2
        \begin{column}{0.33\textwidth}
            \begin{figure}
                \centering
                \includegraphics[width=1.1\columnwidth]{Bilder/average_models/Avg_OHE_OHE_5000E_100BS_1L_1C_200P_1500T_D/Transition_Probability_Matrix;_t=1,_DF=0.5.png}
            \end{figure}
            \begin{center}
                {\large german, \onehot{}} %:\\$ \mu = 7.3$, $ \sigma = 2.0 $
            \end{center}
        \end{column}
        % SPALTE 3
        \begin{column}{0.33\textwidth}
            \begin{figure}
                \centering
                \includegraphics[width=1.1\columnwidth]{Bilder/average_models/Avg_W2V_W2V_5000E_100BS_1L_1C_200P_1500T_D/Transition_Probability_Matrix;_t=1,_DF=0.5.png}
            \end{figure}
            \begin{center}
                {\large german, word vector} %:\\$ \mu = 14.0$, $ \sigma = 2.1 $
            \end{center}
        \end{column}
    \end{columns}
% --------------------------------------
\mynote{
\begin{itemize}
    \item Matrices are $ 10 \times 10 $, so we display them
    \item[i] On the left there is the ground truth of the book illustrated, in the middle a \onehot{} model and on the right the word vector equivalent. The latter shows no learning at all, whereas similarities to the ground truth are recognizable in the middle.
    \item[i] The picture is similar for the english book
\end{itemize}
}
\end{frame}


% --- Results – Averaging models 2/3 -----------------------------------

\begin{frame}{Results – Averaging models}
    \begin{itemize}
        \item<+-> Accuracy of these models is measured by mean and standard deviation:
    \end{itemize}
    \begin{table}
        \centering
%        \caption{Averaged means of the difference between prediction and ground truth. As expected, the word vector versions have higher scores than the \onehot{} counterpart. These values can't be used for a comparison with \tabref{\ref{tab: text model versions and metrics}} because the underlying metrics are different.}
        \begin{tabular}{lrc}
            \toprule
            Version					& Mean $ \mu $		& Standard deviation $ \sigma $ \\
            \midrule
            german, \onehot{} 		& $ 7.3 $	& $ 2.0 $ \\% 0.08200
            german, word vector		& $ 14.0 $	& $ 2.1 $ \\% 0.74139
            english, \onehot{}		& $ 8.1 $	& $ 3.3 $ \\% 0.10021
            english, word vector	& $ 10.2 $	& $ 3.6 $ \\% 0.77522
            \bottomrule
        \end{tabular}
        \label{tab: avg model versions and metrics}
    \end{table}
    \hfill\notsoimportant{Mean and standard deviation in $ 10^{-2} $}
    \begin{itemize}
%        \item<+-> Again \onehot{s} outperform word vectors
        \item<+-> Sadly, the outcome of word vector models is quite bad again % (word vectors might be closer to real input signals of the hippocampus)
        \item<+-> But the \onehot{s} seem to grasp the grammatical structure (bar and matrix plot)
    \end{itemize}
% --------------------------------------
\mynote{
\begin{itemize}
    \item The accuracy of these models is measured by mean and standard deviation 	
    \item Sadly, the outcome of word vector models is not worth mentioning it again. This is in particular unsatisfactory because word vectors might be closer to real signals.
    \item But \onehot{} approaches seem to grasp the grammatical structure, which is justified by the bar and matrix plots   
    \item[x] FALLS JEMAND FRAGT Anderes Maß, es hier explizit die verschiedenen Zeilen mit der Ground truth verglichen wurden, Außerdem hat man weniger Spalten bzw. Zustände im Allgemeinen vorliegen, sodass man bspw. mit der Angabe der Standardabweichung auch etwas anfangen kann
    \item[x] FALLS JEMAND FRAGT
    \begin{itemize}
        \item[x] „Mean“ bedeutet: GT - SR, then row-wise mean, finally mean of means
        \item[x] „Std. deviation“ GT - SR, then row-wise std. dev., finally std. dev of std. devs.
    \end{itemize}
\end{itemize}
}
\end{frame}   % (\chapter{})
\cleardoublepage
%% ... more chapters ....
\chapter{Conclusion}
Since no data from valid neural scans researching the same or a similar topic was provided, the project is heavily theory based (in comparison to~\cite{StBoGe17HPM}), which makes the interpretation of the results a priori not easy because a quantitative and qualitative frame of reference is missing. There were sharp results produced in \secreff{sec: first model and architecture MR}, but they are too artificial to draw relevant conclusions regarding the objective and neuroscientist won't collect data as clear.

While trying to reproduce them \ie getting as close as possible, many architectures were unsuccessfully tested and evaluated as mentioned in \appref{ch: additional configurations}. Therefore, the process of finding a proper one consumed many weeks with discussions between my advisor and me. Maybe, the goals were slightly too ambitious. Additionally, they more or less led to the same results covered in this thesis, especially in \secreff{sec: text based models and architecture}. Although, the values calculated in \tabref{\ref{tab: text model versions and metrics}} are relatively low and close to $ 0 $, which means a perfect fit according to \secreff{sec: metric}, the metric $ d_A $ itself isn't justified for more than internal comparisons of the configurations. It was developed to have a sensible measure on the results because plots of high dimensional sparse matrices, which are just monochromatic squares, aren't convincing. From the figures in \tabref{\ref{tab: text model versions and metrics}} \& \tabref{\ref{tab: avg model versions and metrics}} can be deduced that models training with word vectors have an unsatisfactory performance in comparison to their equivalents working with \onehot{s}. This is unsatisfying for two reasons: Firstly, getting the approach working costed much time because the implementation of the mechanics \spacy{} offers and integrating them into the concept of the \cognitiveroom{} were the most elaborate part in the process of building the framework and secondly, because the input received by the hippocampus is probably closer to a word vector than to a \onehot{}. Translating it into a neuroscience behavior, it can be compared with receiving plenty of signals as input against processing one (strong) activation from another cell.

By presenting the topic in front of my colloquium, further approaches were gathered. One of them, the idea of averaging the predictions of one word class and analyzing the outcome, paid off (s. \secreff{subsubsec: average approach} \& \secreff{sec: average approach}). The results found there can function as addition to the plain word models because they prove that these models can grasp the grammatical structure. Therefore, they can provide visual feedback even in large dimensional contexts and by calculating the ground truth distribution there is a useful reference.

But nevertheless, the \onehot{} variant can serve as foundation for further (minor) research, for instance developing a better metric to have a mathematical notion of encoding a good and objective value or tweaking the learning with word vectors due to the better compatibility with hippocampal functions. Of great value would be collected data from an analogue survey conducted by neuroscientist \ie analyzing the activity of the hippocampus, the place and grid cells while participants process new pieces of language. Then is a viable environment, as in~\cite{StBoGe17HPM}, given and the theory may be expanded to language related topics as it is to spatial navigation.
   % Conclusion
\cleardoublepage

\appendix
\cleardoublepage
\chapter{Additional Configurations} \label{ch: additional configurations}

To draw a full picture, plenty of approaches which had the goal to improve the results will be mentioned in this chapter. Sadly, no one changed the outcome by any means. Facing this was an enormous obstacle while researching. Some of them will be presented shortly in this chapter. In all cases, it is obvious that these configurations were dead ends.

% --------------------------------------
\clearpage
\section{Multiple hidden layers} \label{subsubsec: multiple hidden layers}
Different numbers of layers ranging from $ 1 $ to $ 100 $ were tested, some example results will be depicted.
\newcommand{\hh}{0.365\textheight}
%
\begin{figure}[H]
	\centering
	\subcaptionbox{\gls{sr} of a model using English and \onehot{s} with $ 40 $ hidden layers.}{
		\includegraphics[height=\hh]{Bilder/chapter4/additional_configurations/OHE_OHE_4000E_100BS_40L_1C_200P_1500T_J/Transition_Probability_Matrix;_t=1,_DF=0.5.png}
	}
	\hfill
	\subcaptionbox{\gls{mds} of the matrix in (a).}{
		\includegraphics[height=\hh]{Bilder/chapter4/additional_configurations/OHE_OHE_4000E_100BS_40L_1C_200P_1500T_J/MDS_of_Transition_Probability_Matrix;_t=1,_DF=0.5.png}
	}
	\caption{Although mentioned that transition probability matrices don't show anything if too many words are used, they can be used sometimes to detect failure. The reference, at least for the \gls{mds}, is illustrated in \figref{\ref{fig: w2w model gt en}}. The value of the metric is $ 0.50 $, the equivalent with one layer achieves $ 0.10 $. This could imply that more layers hamper learning.}
	\label{fig: text model en ohe 40L}
\end{figure}
%
\clearpage
\begin{figure}[H]
	\centering
		\subcaptionbox{\gls{sr} of a model using English and \onehot{s} with $ 10 $ hidden layers.}{
			\includegraphics[height=\hh]{Bilder/chapter4/additional_configurations/OHE_OHE_7000E_100BS_10L_1C_200P_1500T_J/Transition_Probability_Matrix;_t=1,_DF=0.5.png}
		}
%	\subcaptionbox{\gls{mds} of a model using english and \onehot{s} with $ 2 $ hidden layers.}{
%		\includegraphics[height=\hh]{Bilder/chapter4/additional_configurations/OHE_OHE_4000E_100BS_2L_1C_200P_1500T_J/MDS_of_Transition_Probability_Matrix;_t=1,_DF=0.5.png}
%	}
	\hfill
	\subcaptionbox{\gls{mds} of a model using English and \onehot{s} with $ 2 $ hidden layers.}{
		\includegraphics[height=\hh]{Bilder/chapter4/additional_configurations/OHE_OHE_4000E_100BS_2L_1C_200P_1500T_J/MDS_of_Transition_Probability_Matrix;_t=1,_DF=0.5.png}
	}
%	\subcaptionbox{\gls{sr} of a model using english and \onehot{s} with $ 10 $ hidden layers.}{
%		\includegraphics[height=\hh]{Bilder/chapter4/additional_configurations/OHE_OHE_7000E_100BS_10L_1C_200P_1500T_J/Transition_Probability_Matrix;_t=1,_DF=0.5.png}
%	}
	\caption{As mentioned before, more layers result in a worse \gls{sr}. Already one additional hidden layer lowers the metric. The model in (a) reaches $ 0.22 $, whereas with one hidden layer the value is $ 0.10 $. The network in (b) was configured with $ 2 $ hidden layers and the successor representation looks indistinguishable from one training with $ 40 $ (\figref{\ref{fig: text model en ohe 40L}}).}
\end{figure}

% --------------------------------------

\clearpage
\section{Many epochs and multiple hidden layers}
The example outputs stem from a model which was trained with sixfold epochs and $ 40 $ hidden layers.
\begin{figure}[H]
	\centering
		\subcaptionbox{\gls{sr} of a model using English, \onehot{s}, $ 25,000 $ epochs and $ 40 $ layers.}{
		\includegraphics[height=\twocolpicheight]{Bilder/chapter4/additional_configurations/OHE_OHE_25000E_100BS_40L_1C_200P_1500T_J/Transition_Probability_Matrix;_t=1,_DF=0.5.png}
	}
	\hfill
	\subcaptionbox{\gls{mds} of a model using English, word vectors, $ 25,000 $ epochs and $ 40 $ layers}{
		\includegraphics[height=\twocolpicheight]{Bilder/chapter4/additional_configurations/W2V_W2V_25000E_100BS_40L_1C_200P_1500T_J/MDS_of_Transition_Probability_Matrix;_t=1,_DF=0.5.png}
	}
%	\caption{The only conclusion to draw from this plot of two different models is that additional epochs might lead to a full degeneration of the results if word vectors are used for training. The \onehot{} analogue shows no mismatch to them of \secreff{subsubsec: multiple hidden layers}.}
	\caption{Trying a combination of numerous epochs in combination with a relative high number of hidden layers leaves a degenerated result.}
\end{figure}

% --------------------------------------

\clearpage
\section{Using word vectors to learn a \onehot{}}
This configuration is a combination of the two mainly used in the thesis. It uses word vectors as input and \onehot{s} as output \ie heterogeneous structured training data.
\begin{figure}[H]
	\centering
	\subcaptionbox{English, word vector to \onehot{} as \gls{mds}.}{
		\includegraphics[width=\twocolpicwidth]{Bilder/chapter4/additional_configurations/W2V_OHE_5000E_100BS_1L_1C_200P_1500T_D/MDS_of_Transition_Probability_Matrix;_t=1,_DF=0.5.png}
	}
	\hfill
	\subcaptionbox{German, ground truth \gls{mds} of word to word transitions.}{
		\includegraphics[width=\twocolpicwidth]{Bilder/chapter4/W2W/ground_truths/MDS_D_200pages_1500T_words.png}
	}
	\caption{\gls{mds} plot of a model using german training data and word vectors as input to learn an \onehot{}. The results lack characteristics to draw sensible conclusions from. Though, it is possible to calculate the metric for the configuration: $ 0.47 $. By comparing it with \tabref{\ref{tab: text model versions and metrics}}, it is situated between its full \onehot{} and word vector relatives.}
\end{figure}

% --------------------------------------

\clearpage
\section{Multiplying the training data} \label{subsubsec: multiplying training data}
The goal of multiplying the training data \ie concatenating the training data $ n $ times with itself, was to have the opportunity to process the training data more often during one epoch.
\begin{figure}[H]
	\centering
	\subcaptionbox{German, \onehot{s} with $ 5 $ concatenations.}{
		\includegraphics[width=\twocolpicwidth]{Bilder/chapter4/additional_configurations/OHE_OHE_5000E_100BS_1L_5C_200P_1500T_D/MDS_of_Transition_Probability_Matrix;_t=1,_DF=0.5.png}
	}
	\hfill
	\subcaptionbox{German, word vectors with $ 5 $ concatenations.}{
		\includegraphics[width=\twocolpicwidth]{Bilder/chapter4/additional_configurations/W2V_W2V_5000E_100BS_1L_5C_200P_1500T_D/MDS_of_Transition_Probability_Matrix;_t=1,_DF=0.5.png}
	}
	\caption{Concatenating the training data $ 5 $ times with itself doesn't change outcomes (compare \figref{\ref{fig: text model cumulativ mds plots}}). This impression is fortified by the metrics both models achieve: $ 0.14 $ for \onehot{s} and $ 0.72 $ with word vectors ($ 0.08 $ and $ 0.74 $ without respectively, \tabref{\ref{tab: text model versions and metrics}}).}
\end{figure}

% --------------------------------------

\clearpage
\section{Calculating high time steps}
One idea was to calculate high time steps of the \gls{sr} hoping the irregularities even out in distant future.
\begin{figure}[H]
	\centering
	\subcaptionbox{\gls{mds} of a \gls{sr} with $ t = 20 $. German and \onehot{s} were used during training.}{
		\includegraphics[width=\twocolpicwidth]{Bilder/chapter4/additional_configurations/hohes_t/OHE_OHE_5000E_100BS_1L_1C_200P_1500T_D/MDS_of_SR,_t=20,_DF=0.5.png}
	}
	\hfill
	\subcaptionbox{\gls{mds} of a \gls{sr} with $ t = 50 $. German and \onehot{s} were used during training.}{
		\includegraphics[width=\twocolpicwidth]{Bilder/chapter4/additional_configurations/hohes_t/OHE_OHE_5000E_100BS_1L_1C_200P_1500T_D/MDS_of_SR,_t=50,_DF=0.5.png}
	}
	\caption{High time steps also don't facilitate progress since no evident structure is recognizable within the plots. A comparison with the ground truth wouldn't add additional insights too because the underlying matrices can't be interpreted as transition probability matrices.}
	\label{fig: high time steps ohe}
\end{figure}
%
\clearpage
\begin{figure}[t]
	\centering
	\subcaptionbox{\gls{mds} of a \gls{sr} with $ t = 20 $. German and word vectors were used during training.}{
		\includegraphics[width=\twocolpicwidth]{Bilder/chapter4/additional_configurations/hohes_t/W2V_W2V_5000E_100BS_1L_1C_200P_1500T_D/MDS_of_SR,_t=20,_DF=0.5.png}
	}
	\hfill
	\subcaptionbox{\gls{mds} of a \gls{sr} with $ t = 50 $. German and word vectors were used during training.}{
		\includegraphics[width=\twocolpicwidth]{Bilder/chapter4/additional_configurations/hohes_t/W2V_W2V_5000E_100BS_1L_1C_200P_1500T_D/MDS_of_SR,_t=50,_DF=0.5.png}
	}
	\caption{As before in \figref{\ref{fig: high time steps ohe}} no structure is recognizable to do further research on. Results relying on word vectors again seem to be very labile and one dimensional.}
\end{figure}

% --------------------------------------

\clearpage
\section{Predict only the most frequent words}
Similar to \secreff{subsubsec: multiplying training data}, the most frequent words of the text are seen more often by the network. Hence, it might be able to learn these inputs better than ordinary ones.
\begin{figure}[H]
	\centering
	\subcaptionbox{German with word vectors. \gls{mds} of the $ 40 $ most frequent words.}{
		\includegraphics[width=\twocolpicwidth]{Bilder/chapter4/additional_configurations/MostFrequentWords_4000E_100BS_1L_1C_200P_1500T_D/MDS_of_Transition_Probability_Matrix;_t=1,_DF=0.5.png}
	}
	\hfill
	\subcaptionbox{Same \gls{mds} plot as in (a) but annotated with \postag{s} instead of words.}{
		\includegraphics[width=\twocolpicwidth]{Bilder/chapter4/additional_configurations/MostFrequentWords_4000E_100BS_1L_1C_200P_1500T_D/ud_pos_tag_annotated.png}
	}
	\caption{The same model with a german data set and word vectors as in \secreff{sec: text based models and architecture} was trained. Predictions were limited to the $ 40 $ most frequent words. For a better overview the second plot was labeled with the corresponding \postag{s}.}
\end{figure}

\cleardoublepage
\chapter{Cluster plots of word to word models} \label{ch: appendix text model}
Visualizing matrices training with more than 500 words doesn't make sense, because the result is high dimensional and sparse. To get at least some visual feedback, \gls{mds} plots were calculated. To avoid a crowded picture, a compromise had to be made: depicting solely the \postag{} provides a manageable overview but information on single words gets lost.
\begin{figure}[H]
	\centering
		\subcaptionbox{German, ground truth \gls{mds} of word to word transitions.}{
			\includegraphics[width=\twocolpicwidth]{Bilder/chapter4/W2W/ground_truths/MDS_D_200pages_1500T_words.png}
		}
		\hfill
		\subcaptionbox{English, ground truth \gls{mds} of word to word transitions.\label{fig: w2w model gt en}}{
			\includegraphics[width=\twocolpicwidth]{Bilder/chapter4/W2W/ground_truths/MDS_J_200pages_1500T_words.png}
		}
	\caption[]{\gls{mds} of ground truth using german and english training data.}
	\label{fig: text model gt de en mds}
\end{figure}
\begin{figure}
	\centering
		\subcaptionbox{German, \gls{mds} of learned \gls{sr} using \onehot{s}. Metric: $ 0.08 $.}{
			\includegraphics[width=\twocolpicwidth]{Bilder/chapter4/W2W/plots/OHE_OHE_5000E_100BS_1L_1C_200P_1500T_D/_epoch-4000/MDS_of_Transition_Probability_Matrix;_t=1,_DF=0.5.png}
		}
		\hfill
		\subcaptionbox{English, \gls{mds} of learned \gls{sr} using \onehot{s}. Metric: $ 0.1 $.}{
			\includegraphics[width=\twocolpicwidth]{Bilder/chapter4/W2W/plots/OHE_OHE_4000E_100BS_1L_1C_200P_1500T_J/MDS_of_Transition_Probability_Matrix;_t=1,_DF=0.5.png}
		}
		\\
		\subcaptionbox{German, \gls{mds} of learned \gls{sr} using word vectors. Metric: $ 0.74 $.}{
			\includegraphics[width=\twocolpicwidth]{Bilder/chapter4/W2W/plots/W2V_W2V_5000E_100BS_1L_1C_200P_1500T_D/_epoch-4000/MDS_of_Transition_Probability_Matrix;_t=1,_DF=0.5.png}
		}
		\hfill
		\subcaptionbox{English, \gls{mds} of learned \gls{sr} using word vectors. Metric: $ 0.78 $.}{
			\includegraphics[width=\twocolpicwidth]{Bilder/chapter4/W2W/plots/W2V_W2V_5000E_100BS_1L_1C_200P_1500T_J/_epoch-4000/MDS_of_Transition_Probability_Matrix;_t=1,_DF=0.5.png}
		}
	\caption{The \onehot{} models show some resemblance with the ground truth in \figref{\ref{fig: text model gt de en mds}}. The disappointing results of word vectors is not just visible by the different shape the dots occupy but also by their number, much less are visible \ie many are mapped onto each other. Hence the network produces the same output for different inputs.}
	\label{fig: text model cumulativ mds plots}
\end{figure}
   % appendix A
\cleardoublepage
\chapter{Barplots of the average approach} \label{ch: appendix average approach}

The barplots illustrated here stem from an \onehot{} model using a german book to collect the training data. It achieved the best value ($ 7.3 \cdot 10^{-2} $) of the four tested models. The scores are listed in \tabref{\ref{tab: avg model versions and metrics}}. If a \postag{} doesn't appear in the plots \ie having no green or blue bar, this means that it doesn't succeed the depicted \postag{}. The explanation of the \postag{s} can be found in \tabref{\ref{tab: ud pos tags}}.
\begin{figure}[H]
	\centering
		\subcaptionbox{Averaged transitions of all \texttt{ADP}s compared to the ground truth.}{
			\includegraphics[width=\twocolpicwidth]{Bilder/chapter4/Barplots/Avg_OHE_OHE_5000E_100BS_1L_1C_200P_1500T_D/_epoch-4000/Combined_Barplot_ADP_S.png}
		}
		\hfill
		\subcaptionbox{Averaged transitions of all \texttt{ADJ}s compared to the ground truth.}{
			\includegraphics[width=\twocolpicwidth]{Bilder/chapter4/Barplots/Avg_OHE_OHE_5000E_100BS_1L_1C_200P_1500T_D/_epoch-4000/Combined_Barplot_ADJ_F.png}
		}
	\caption{Barplot of \texttt{ADP} and \texttt{ADJ}.}
\end{figure}
\begin{figure}[H]
	\centering
	\subcaptionbox{Averaged transitions of all \texttt{DET}s compared to the ground truth.}{
		\includegraphics[width=\twocolpicwidth]{Bilder/chapter4/Barplots/Avg_OHE_OHE_5000E_100BS_1L_1C_200P_1500T_D/_epoch-4000/Combined_Barplot_DET_S.png}
	}
	\hfill
	\subcaptionbox{Averaged transitions of all \texttt{ADV}s compared to the ground truth.}{
		\includegraphics[width=\twocolpicwidth]{Bilder/chapter4/Barplots/Avg_OHE_OHE_5000E_100BS_1L_1C_200P_1500T_D/_epoch-4000/Combined_Barplot_ADV_F.png}
	}
	\caption{Barplot of \texttt{DET} and \texttt{ADV}.}
\end{figure}
\begin{figure}[H]
	\centering
	\subcaptionbox{Averaged transitions of all \texttt{AUX}s compared to the ground truth.}{
		\includegraphics[width=\twocolpicwidth]{Bilder/chapter4/Barplots/Avg_OHE_OHE_5000E_100BS_1L_1C_200P_1500T_D/_epoch-4000/Combined_Barplot_AUX_F.png}
	}
	\hfill
	\subcaptionbox{Averaged transitions of all \texttt{NOUN}s compared to the ground truth.}{
		\includegraphics[width=\twocolpicwidth]{Bilder/chapter4/Barplots/Avg_OHE_OHE_5000E_100BS_1L_1C_200P_1500T_D/_epoch-4000/Combined_Barplot_NOUN_F.png}
	}
	\caption{Barplot of \texttt{AUX} and \texttt{NOUN}.}
\end{figure}
\begin{figure}[H]
	\centering
	\subcaptionbox{Averaged transitions of all \texttt{VERB}s compared to the ground truth.}{
		\includegraphics[width=\twocolpicwidth]{Bilder/chapter4/Barplots/Avg_OHE_OHE_5000E_100BS_1L_1C_200P_1500T_D/_epoch-4000/Combined_Barplot_VERB_S.png}
	}
	\hfill
	\subcaptionbox{Averaged transitions of all \texttt{REST}s compared to the ground truth.}{
		\includegraphics[width=\twocolpicwidth]{Bilder/chapter4/Barplots/Avg_OHE_OHE_5000E_100BS_1L_1C_200P_1500T_D/_epoch-4000/Combined_Barplot_REST_S.png}
	}
	\caption{Barplot of \texttt{VERB} and \texttt{REST}.}
\end{figure}
\begin{figure}[H]
	\centering
	\subcaptionbox{Averaged transitions of all \texttt{PART}s compared to the ground truth.}{
		\includegraphics[width=\twocolpicwidth]{Bilder/chapter4/Barplots/Avg_OHE_OHE_5000E_100BS_1L_1C_200P_1500T_D/_epoch-4000/Combined_Barplot_PART_S.png}
	}
	\hfill
	\subcaptionbox{Averaged transitions of all \texttt{PRON}s compared to the ground truth.}{
		\includegraphics[width=\twocolpicwidth]{Bilder/chapter4/Barplots/Avg_OHE_OHE_5000E_100BS_1L_1C_200P_1500T_D/_epoch-4000/Combined_Barplot_PRON_F.png}
	}
	\caption{Barplot of \texttt{PART} and \texttt{PRON}.}
\end{figure}
   % appendix B
\cleardoublepage
\chapter{Training parameters} \label{ap: parameters}
%
The values of the parameters in \tabref{\ref{tab: training parameters}} were used to produce the results with the presented framework. For the outcomes in \appref{ch: additional configurations}, the numbers differ \eg more \texttt{epochs}, a higher \texttt{nmb\_hidden\_layers} or \texttt{nmb\_concatenations}. The exact factor is mentioned there.
\begin{table}[H]
	\centering
	\caption{Training parameters of the network.}
	\begin{tabular}{ll}
		\toprule
		Parameter & Value \\
		\midrule
		Learning rate \texttt{lr} & $ 0.1 $ \\
		\texttt{epochs} & $ 4000 $ \\
		\texttt{batch\_size} & $ 100 $ \\
		\texttt{pages} & $ 200 $ \\
		\texttt{nmb\_tokens} & $ 1500 $ \\
		\texttt{nmb\_hidden\_layers} & $ 1 $ \\
		\texttt{nmb\_concatenations} & $ 1 $ \\
		\texttt{book\_name} & $ 0 $ (german), $ 1 $ (english)\\
		\bottomrule
	\end{tabular}
	\label{tab: training parameters}
\end{table}

\cleardoublepage


%% Do not change, auto-generated lists of figures, tables and literature %%
% Glossar
\printunsrtglossary[type=abbreviations]
\cleardoublepage

% List of figures
\addcontentsline{toc}{chapter}{\listfigurename}
\listoffigures
\cleardoublepage

% List of tables
\addcontentsline{toc}{chapter}{\listtablename}
\listoftables
\cleardoublepage

% Literature list
% %CONFIG:
%\selectlanguage{german}{\addcontentsline{toc}{chapter}{\bibname}}
\selectlanguage{english}{\addcontentsline{toc}{chapter}{\bibname}}

\printbibliography
\end{document}
