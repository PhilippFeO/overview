%%%%%%%%%%%%%%%%%%%%%%%%%%%%%%%%%%%%%%%%%%%%%%%%%%%%%%%%%%%%%%%%%%%%%%%%%%%%%%%%%%%%%%%%%%%%%%%%%%%%
% 
% PRL presentation template
% based on https://github.com/FAU-AMMN/fau-beamer by Tim Roith
% Adaptation done by:
% Linda Schneider
% Dalia Rodriguez-Salas
% Noah Maul
%
% This program can be redistributed and/or modified under the terms
% of the GNU Public License, version 2.
%
%%%%%%%%%%%%%%%%%%%%%%%%%%%%%%%%%%%%%%%%%%%%%%%%%%%%%%%%%%%%%%%%%%%%%%%%%%%%%%%%%%%%%%%%%%%%%%%%%%%%

%\documentclass[handout, notes, t]{beamer}
%\documentclass[notes=only, t]{beamer}
\documentclass[handout, t]{beamer}
%\documentclass[t]{beamer}

%\setbeameroption{show only notes}

\usepackage[utf8]{inputenc}
\usepackage[T1]{fontenc}

%%%%%%%%%%%%%%%%%%%%%%%%%%%%%%%%%%%%%%%%%%%%%%%%%%%%%%%%%%%%%%%%%%%%%%%%%%%%%%%%%%%%%%%%%%%%%%%%%%%%
%% VON MIR
%%%%%%%%%%%%%%%%%%%%%%%%%%%%%%%%%%%%%%%%%%%%%%%%%%%%%%%%%%%%%%%%%%%%%%%%%%%%%%%%%%%%%%%%%%%%%%%%%%%%
\usepackage{enumitem}
\usepackage{booktabs}
\usepackage{float}
\usepackage{nameref}
\newcommand{\Emph}[1]{\textbf{#1}}
\newcommand{\mynote}[1]{\note{\huge{#1}}}
\newcommand{\spacy}{\texttt{spacy}}
\newcommand{\onehot}[1]{$ 1 $-hot-encoded vector{#1}}

\newcommand{\os}[1]{\onslide<+->{#1}}
\newcommand{\Os}[2]{\onslide<#1->{#2}}

\newcommand*{\figref}[1]{\figurename~#1}

%\setbeamercovered{transparent=0}

%\usepackage{xspace}
%\newcommand{\ie}{i.\,e.,\xspace}
%\newcommand{\eg}{e.\,g.,\xspace}

%%%%%%%%%%%%%%%%%%%%%%%%%%%%%%%%%%%%%%%%%%%%%%%%%%%%%%%%%%%%%%%%%%%%%%%%%%%%%%%%%%%%%%%%%%%%%%%%%%%%
%% STYLE
%%%%%%%%%%%%%%%%%%%%%%%%%%%%%%%%%%%%%%%%%%%%%%%%%%%%%%%%%%%%%%%%%%%%%%%%%%%%%%%%%%%%%%%%%%%%%%%%%%%%

% the possible options for the 'i2beamer' package are:
% - 'english' (if the slides are in english) 
% - 'wide'    (if the slides should use a 16:9 aspect ratio)
% - 'logo'    (if the logo of the Programming Systems Group should be included)
% - 'plain'   (if the background image of the title page should be omitted)
% - Faculty shorts: nat, phil, med, tf, rw, wiso, jura (no faculty short results in FAU)

\RequirePackage[wide,logo,tf,english]{sty/beamer}

% VON MIR
% Nachfolgende Stichpunkte schimmern nicht in grau durch
% Muss NACH \RequirePackage{.../beamer} stehen
\setbeamercovered{transparent=0}

%%%%%%%%%%%%%%%%%%%%%%%%%%%%%%%%%%%%%%%%%%%%%%%%%%%%%%%%%%%%%%%%%%%%%%%%%%%%%%%%%%%%%%%%%%%%%%%%%%%%
%% Bibliography
%%%%%%%%%%%%%%%%%%%%%%%%%%%%%%%%%%%%%%%%%%%%%%%%%%%%%%%%%%%%%%%%%%%%%%%%%%%%%%%%%%%%%%%%%%%%%%%%%%%%
\defbibheading{bibliography}{}
\addbibresource[label=primary]{references.bib}
\nocite{*}

%%%%%%%%%%%%%%%%%%%%%%%%%%%%%%%%%%%%%%%%%%%%%%%%%%%%%%%%%%%%%%%%%%%%%%%%%%%%%%%%%%%%%%%%%%%%%%%%%%%%
%% INFO
%%%%%%%%%%%%%%%%%%%%%%%%%%%%%%%%%%%%%%%%%%%%%%%%%%%%%%%%%%%%%%%%%%%%%%%%%%%%%%%%%%%%%%%%%%%%%%%%%%%%

\title{\parbox{\textwidth}{The hippocampus and language: Word to word prediction in terms of the successor representation}}
%\subtitle{Optional subtitle of document}

\author{Philipp Rost}
\institute[Department Informatik]{Friedrich-Alexander-Universität Erlangen-Nürnberg\\ Department Informatik}
%\institute{Friedrich-Alexander-Universität Erlangen-Nürnberg}

\date{\today}

%%%%%%%%%%%%%%%%%%%%%%%%%%%%%%%%%%%%%%%%%%%%%%%%%%%%%%%%%%%%%%%%%%%%%%%%%%%%%%%%%%%%%%%%%%%%%%%%%%%%
%% DOCUMENT
%%%%%%%%%%%%%%%%%%%%%%%%%%%%%%%%%%%%%%%%%%%%%%%%%%%%%%%%%%%%%%%%%%%%%%%%%%%%%%%%%%%%%%%%%%%%%%%%%%%%

\begin{document}

% === TITEL FOLIE ===================================

\begin{frame}[plain,c]
  \titlepage
\end{frame}

% === AGENDA ===================================

\begin{frame}
    \frametitle{Agenda}
    \tableofcontents
% --- NOTIZEN -----------------------------------
\mynote{
    \begin{itemize}
        \item[ü] The agenda consists of five sections: Introduction, the theoretical background, presenting the framework I have developed, my results and a conclusion 
    \end{itemize}
}
\end{frame}

% === INTRODUCTION ===================================

\section{Introduction}
\begin{frame}{Introduction}
    \begin{itemize}
    	\item<+-> Understanding the human brain is a challenge as old as science itself
    	\item<+-> Currently available technology as a metaphor (from abacus to computer)
    	\item<+-> Projects exist researching the brain as a whole but also for distinct parts, e.g. the hippocampus
    	\item<+-> Goal: Expanding the application of the Successor Representation (SR) to language
    	\begin{itemize}
    		\item<5-> supposedly used by the hippocampus to predict following states/positions
    	\end{itemize}
    	\item<6-> To reach it, a neural network is trained with samples extracted from two books
    \end{itemize}
% --------------------------------------
\mynote{
\begin{itemize}
    \item[ü] I will start with a short overview
	\item Understanding the human brain is a challenge as old as science itself
	\item Currently available technology serves as a metaphor (from abacus to computer)
    \item There exists projects researching the brain as a whole but also for distinct parts, e.g. the hippocampus
	\item The goal of my thesis is the expansion of the Successor Representation. STICHPUNKT EINBLENDEN Via the SR we can kinda predict future states/positions in an environment, for example our location in a city. The technique was applied beforehand successfully to a spatial environment and shall now be extended to an abstract scenario like language. To put in bluntly, is it possible to achieve proper (long term) word to word predictions by this technique?
	\item To reach it, a neural network is trained with samples extracted from two books
\end{itemize}
}
\end{frame}

% === CHAPTER ===================================

\chapter{Theoretical Background}

% ======================================

\section{Hippocampus} \label{sec: Hippocampus}
The hippocampus is located in the brain and part of an old area called the archicortex. It is named after the greek word for seahorse, because it has the shape of one (\figref{\ref{wrapfig: Hippocampus and seahorse}}). This brain area can be divided into three parts: the dentate gyrus, the cornu ammonis and the subiculum~\cite{ORFrHa20CCN, GarzorzStark18BN}.

\begin{wrapfigure}{r}{0.35\textwidth}
	\centering
		\includegraphics[width=0.35\textwidth]{Hippocampus_and_seahorse.jpg}
	\caption{Hippocampus and seahorse~\cite{Seress10H}} % TODO Besser formatieren oder entfernen
	\label{wrapfig: Hippocampus and seahorse}
\end{wrapfigure}
The hippocampus plays a fundamental role in forming new memories (not preserving them, which is done across the brain) and is highly capable of learning new information fast. Regarding its functions, one was already mentioned: It is the key area when it comes to establishing new memories. Patients with a damaged hippocampus, therefore lacking this ability, will lose spatial and temporal orientation. Moreover epilepsy, schizophrenia and Alzheimer's disease are connected to this dysfunctional organ~\cite{Trepel17N}.
The hippocampus is also important in emotional contexts because it is an integral unit of the limbic system~\cite{GarzorzStark18BN}. Another task, and for this thesis the most important one, is navigation/orientation, not just in spatial surroundings, but also in an abstract context, called \cognitiveroom{}. Some examples for abstract contexts are: Danger of animals based on their appearance and speed of vehicles based on their weight and engine (\secreff{sec: predictive map theory}). To achieve this skill, two types of cells in the hippocampus are active: place cells and grid cells.
The first one encodes states/positions (one for each cell) and the latter resembles a coordinate system.
\paragraph{Place cells} \label{par: Place cell}
Place cells are irregular distributed across the cognitive room. Their firing is tied to the location of the state, whereby the term location has not always its classic spatial meaning if we navigate in an abstract setting (as mentioned above). The place cell is active in case we encounter the associate state. As seen in \figref{\ref{fig: Rat in maze}}, different place cells (each is color coded) fire at different positions in the parkour \eg turquoise is undoubtedly related to the first arch, meaning its activity spikes while the rat passes by. The remark of the thesis lies on place cells.
\begin{figure}
	\centering
		\includegraphics{Ortszelle_Beispiel.png}
	\caption{Activity pattern of color encoded place cell across a maze. Each place cell is exactly related to one distinct position of the corresponding environment \eg turquoise to the first arch. Its activity spikes if the rat walks along the arch~\cite{Stuartlayton13}.}
	\label{fig: Rat in maze}
\end{figure}
%This means in a spatial scenario, where we walk around in a square \TODO[fancyline]{Bild in GeoGebra erstellen mit Quadrat, Brunnen, Baum..., daneben Aktvierungsmuster der Ortszellen} (i.e. with a fountain), it fires irregularly and always then if we are at the position the place cell encodes. In case it resembles the fountain it will fire if we are close to it. \\
\paragraph{Grid cells}
This type of cell can be found in the entorhinal region and satisfies a more general purpose. They are regularly distributed and form a triangular lattice (\figref{\ref{fig: Grid cells}}). It provides raw spatial information in terms of a metric or distance measure the hippocampus integrates with the place cells~\cite{ORFrHa20CCN, BellmundEtAl18NC}.
\begin{figure}
	\centering
		\includegraphics[scale=0.25]{Gitterzelle_Beispiel.png}
	\caption{Sketched path of a rat moving in a square, while tracking firing grid cells. As their name suggests, they form a regular lattice over the space. Hence, they act as coordinate system. The information provided by grid cells is combined with that of the place cells to generate a full picture of the surroundings~\cite{Moser15PGM}.}
	\label{fig: Grid cells}
\end{figure}


% ======================================

\section{Predictive map theory} \label{sec: predictive map theory}
To explain the principle of the predictive map theory introduced in~\cite{StBoGe17HPM}, it is necessary to illustrate the concept of a \cognitiveroom{}, mentioned before in \secreff{sec: Hippocampus}. An example is of course a naive navigational task as presented by \etal{Stachenfeld} and similar to the setting of \figref{\ref{fig: Rat in maze}}. The authors even demonstrated that just a topological environment is sufficient to craft a \cognitiveroom{} and apply the predictive map theory.

The concept becomes far more interesting when talking about cognitive rooms founded on experience \ie the speed of vehicles based on weight and engine specifications. This category of a \cognitiveroom{} also fits the topic of the thesis much better, since it aims to model language not a spatial environment. An illustrating example can be found in \figref{\ref{fig: vehicles cognitive room}}. For instance, a ``sports car'' might be rather lightweight but has plenty of horse power.
\begin{figure}
	\centering
		\includegraphics[width=0.4\textwidth]{Fahrzeuge_Cognitive_Room.jpeg}
	\caption{Exemplary \cognitiveroom{} of vehicles according to their weight and engine power. An unknown car can be placed easily in the environment given the two parameters because there are already established place cells acting as abstract waypoints (the depicted cars) to support the orientation \ie finding its place on the map. By doing so, it is immediately possible to derive information about the appearance of the automobile~\cite{BellmundEtAl18NC}.}
	\label{fig: vehicles cognitive room}
\end{figure}
By using these two characteristics, the \cognitiveroom{} has the shape of a $ 2d $-plane. For instance, while reading about an alien car, it is immediately possible to compare it with different well-known vehicles and draw conclusions about its shape since the \cognitiveroom{} has enough information to position the car within it. All these decisions of placing new objects in an appropriate context is done by place and grid cells (\secreff{sec: Hippocampus}). Expanding the example by the firing of cells results in the full illustration given in \figref{\ref{fig: vehicles with place and grid cells}}.
\begin{figure}
	\centering
		\includegraphics[width=1.\textwidth]{Fahrzeuge_Ortszelle_Gitterzelle.jpeg}
	\caption{Left: \cognitiveroom{} of vehicles according to weight and horse power. Middle: Firing pattern of place cells crafting the \cognitiveroom{} \ie the boundaries in \figref{\ref{fig: vehicles cognitive room}}. Right: Corresponding lattice of grid cells.~\cite{BellmundEtAl18NC}}
	\label{fig: vehicles with place and grid cells}
\end{figure}

%
% Paul bezieht die „Prdictive Map theory“ lediglich „auf die Autos und Tiere“
%

%In case of encountering a unknown species the predictive map becomes active in terms of firing place cells encoding animals sharing the same features and thus evaluating the potential danger. In these terms a place cell doesn't encode the current state but a future state because the new living being will be cataloged next to the known ones. So, the cognitive room exhibits a predictive property.
%
%We can, for instance, apply it to the scenario of the rat passing a maze as seen in Fig. \ref{fig: Rat in maze}. In par. \nameref{par: Place cell} was mentioned that a place cell resembles the current position. By the new perspective follows that the cell fires beforehand~\eg again in terms of the turquoise cell: This cell is now contemplated active if the rat is within the are preceding the arch and aims to enter it.
%
%Furthermore, the authors argue that


% ======================================

\section{Successor Representation} \label{sec: SR}
According to \etal{Stachenfeld}, our behavior in an open spatial environment, or in general in a \cognitiveroom{} \eg a city, follows the predictive map theory introduced in \secreff{sec: predictive map theory}. An active place cell encodes the next/successor state entered by the agent.

To model this or the general setting of predicting future states, the \gls{sr} was developed by a \gls{rl} approach. Furthermore, the \gls{sr} and the predictive map theory go hand in hand. The latter is an application regarding the former: The authors support the proposition that hippocampal mechanics, explained by the predictive map theory, can be described via the \gls{sr}.

% ======================================

\subsection{Mathematical Foundation}
The basis lies largely in \gls{rl}, in formula:
\begin{equation}\label{eq: rl}
	V(s) := E \left[
				\sum_{t=0}^{\infty}
					\gamma^t R(s_t) | s_0 = s
			\right]
\end{equation}
with $ V $ resembling a value function, expressed via the reward function $ R $, which operates on state $ s_t $, encoded by the sum over $ t $, starting in $ s $. $ \gamma \in [0,1] $ serves as a discount factor to control the influence of states reached in distant future. High values permit distal states to play a larger role, whereas smaller values de facto limit the result to neighboring positions\footnote{Short mathematical explanation: $ p^t \xrightarrow[]{t \to \infty} 0 $ for $ p \in [0,1) $, the greater $ p $ the slower happens the approach of the limit. For $ p = 1 $ the sequence is constant. In our case every state is taken into account equally.}. By the reward function is obtained how beneficial the currently visited state $ s_t $ is. % The expected value is taken because there are \wlog plenty of paths starting in $ s $.
After the calculation of $ V $, the function can be decomposed into a more intuitive representation, consisting of a state matrix $ M $, called the \srmat{}, and the known reward function $ R $:
\begin{equation}\label{eq: v with sr-matrix}
	V(s) = \sum_{s'}
				M(s, s') \cdot R(s')
	\text{.}
\end{equation}
The first argument of $ M $ specifies the row, the latter the column. Each cell contains the discounted expected number of times the agent visits state $ s' $ starting from $ s $. Additionally, \etal{Stachenfeld} mention that the \srmat{} can be derived from a transition probability matrix $ T $ for the positions $ s $~\cite{StBoGe17HPM}. Having $ T $, it follows
\begin{equation}\label{eq: sr or m via T}
	M = \sum_{t = 0}^{\infty}
			\gamma^t T^t
	\text{,}
\end{equation}
which is a geometric series and converges for $ \gamma < 1 $ towards
\begin{equation}\label{eq: geo series}
	(I_n - \gamma T)^{-1}
	\text{,}
\end{equation}
where $ I_n $ is the corresponding identity matrix.

Although defining all formulae by infinite sums, it is seamlessly possible to calculate the \srmat{} in \equref{\eqref{eq: sr or m via T}} up to a finite index or starting at an arbitrary $ t $ \ie $ t=1 $. Doing so makes sense in a language environment. The identity matrix would imply that a word can follow itself, which is extremely rare\footnote{Although sentences like ``Ich hoffe, dass das das Richtige ist.'' do occure in german.}. Therefore, the first summand will always be $ \gamma T $, where $ T $ is calculated by a Neural Network (\secreff{ch: framework}). If the indices in \equref{\eqref{eq: sr or m via T}} are altered, the limit of the geometric series no longer applies directly and has to be adjusted by subtracting the first summands from \equref{\eqref{eq: geo series}}. The \gls{sr} and thus the depicted formulas, especially the \srmat{} and the transition probability matrix, are policy dependent. This is reflected by the training data.

By definition, matrix $ M $ reveals all successor states with their particular probability because the summation combines the following positions, which are calculated by exponentiation, into one matrix. By examining a row (\figref{\ref{fig: sr-spalte}}) \eg row $ k $, it is possible to follow all paths starting from state $ k $.

One advantage of the \gls{sr}, \ie describing the model by the \srmat{} $ M $, is its high flexibility regarding the evaluation of different reward functions given by \equref{\eqref{eq: v with sr-matrix}}. The value of a state $ s $ can be calculated in an instant with a different reward function while no relearning is necessary.

\paragraph{\gls{sr} and grid cells}
Although not further discussed in the thesis but an interesting claim of \etal{Stachenfeld} is that the eigenvalue decomposition of the \srmat{} reveals the grid cell structure. They provide supplementary information depicting many examples~\cite{StBoGe17HPM}.

% ======================================

\subsection{Example for the Successor Representation}
This subsection is dedicated to fill the concept of the \gls{sr} and the \srmat{} $ M $ with some intuition. \etal{Stachenfeld} simulated a linear spatial environment built by six states with a simple policy merely consisting of two actions the agent can apply: Going one step to the right or pausing.
%
\paragraph{Rows}
In this scenario, a plot of single rows of $ M $ is shown in \figref{\ref{fig: sr-zeile}}.
\begin{figure}
	\centering
		\includegraphics[width=0.7\linewidth]{Beispiel_SR_Zeile.png}
	\caption{Schematic plot of the rows of state $ s^1 $ and $ s^5 $ respectively. By interpreting the ordinate values as probabilities for transitioning instead of a probability for the current state, it is possible to make assumptions on the future path the agent may take. Hence, matrix $ M $ describes all possible paths. In both cases the policy prefers pausing over changing the state.}
	\label{fig: sr-zeile}
\end{figure}
From the upper half of \figref{\ref{fig: sr-zeile}}, examining the row of $ s^1 $, it is possible to deduce that the agent will most likely remain at its current position, with the values for distal locations disappearing. A different point of view results for the part of $ M(s^5, s^i) $. It is obvious that going backwards is nothing to reckon with since the numbers for transitioning distribute over $ s^5 $ and \texttt{goal} by slightly favoring the former. This behavior was expected by the policy.
%
\paragraph{Columns}
It is also worth analyzing the columns of $ M $, called \emph{place fields} by \etal{Stachenfeld} (\figref{\ref{fig: sr-spalte}}). Having the policy in mind, it is no surprise that the values $ M(s^i, s^5) $ ascend in parallel to the index $ i $. The probability for entering $ s^5 $ grows by approaching it. In addition the plot shows how $ s^5 $ is probably reached best, simply by passing via $ s^3 $ and $ s^4 $. This might seem obvious, but in a more complex \cognitiveroom{} the graph won't look as ordinary and therefore will contain plenty of distributed information. %A spike on $ s^1 $ would counteract it because it means stepping from $ s^1 $ to $ s^5 $ directly (as seen in Fig. \ref{fig: sr-zeile} this is not the case).
\begin{figure}
	\centering
		\includegraphics[width=0.7\textwidth]{Beispiel_SR_Spalte.png}
	\caption{Schematic plot of the $ s^5 $-column depicting how $ s^5 $ is reached by ascending probabilities. It is possible to recapitulate the policy consisting of pausing or taking one step to the right. Entering $ s^5 $ is most likely from $ s^4 $ and $ s^5 $ (due to resting). }
	\label{fig: sr-spalte}
\end{figure}


% ======================================

\section{Multidimensional Scaling}
The goal of \gls{mds} is to calculate a $ m $-dimensional mapping, $ m < n $, of a given point cloud in $ \mathbb{R}^n $ that preserves the original distances as good as possible~\cite{HaTiFr17ESL}. The result of the calculation is unique modulo rotation and scaling. Therefore, it is based on a metric and not exact coordinates. \gls{mds} is used to analyze similarities between the rows of the \srmat{} by determining clusters in the graph.\\
%
The algorithm is simple and only uses basic linear algebra. \gls{mds} works with a distance matrix $ D $, where each entry is equal to $ d_{ij}^2 $, the squared distance between two points $ x_i, \ x_j \in \mathbb{R}^n $, whose coordinates are (in principle) unknown. By double centering $ D $ it is possible to calculate the matrix product $ X^\top X $, where $ X $ bears the coordinates in the desired dimension~\cite{Riess20PA}. Double centering means multiplying by a matrix $ C := I_n - \frac{1}{n}J_n$, where $ J_n $ is a $ n \times n $-matrix of ones:
\begin{equation}
	\underbrace{-\frac{1}{2} CDC}_{B :=} = X^\top X
	\text{,}
\end{equation}
The centering matrix $ C $ has, after a multiplication with a column vector, the same effect of subtracting the mean of all components from the vector itself. \\
In the next step, the $ m $ largest eigenvalues of $ B $ are calculated along with their corresponding eigenvectors. Finally, the $ m $-dimensional coordinates are determined:
\begin{equation}
	X_m = E_m V_m^{1/2}
	\text,
\end{equation}
where $ E_m $ contains the $ m $ eigenvectors and $ V_m $ is a $ m $-dimensional diagonal matrix with the associated eigenvalues.


% ======================================

\section{Metric for quantifying the results} \label{sec: metric}
Throughout the presentation of the results in \chapreff{ch: results} numerous matrices and \gls{mds} plots are used. Sometimes, they give a clarifying visual response, but not in all cases. When comparing different approaches, images lack the needed objectivity and plausible criteria to rate the outcomes. To tackle this issue, a metric was developed to have the possibility to draw objective conclusions.

Since Neural Networks on languages are trained, a measure on the grade of the closeness to the real counterpart is necessary, which is referred by ``ground truth (distribution)'' in the following. The mathematical objects are in both cases squared matrices of dimension $ n \in \mathbb{N} $ built by transposed probability vectors. Nevertheless, the presented mapping is made for $ n \times m $-matrices. Depending on the model type, the ground truth vectors are \onehot{s} or share different fractions across all entries (\secreff{sec: w2w models}).

Hence, the starting positions for the metric are probability vectors. The obvious way to quantify the results is by taking the euclidean norm $ d $ of the difference of the ground truth and the prediction. Consequentially, $ d $ takes values between $ 0 $ and $ \sqrt{2 \cdot n} $ because the maximal difference for each row is $ \sqrt{2} $ and there are $ n $ in total. $ \sqrt{2} $ is derived by the following nonlinear program
\begin{align}
	\mathrm{max} 	& \quad \Vert \vec{x} - \vec{y}\Vert_2 = \sqrt{\sum_{i=1}^{n} (x_i - y_i)^2} \nonumber\\
	\mathrm{s.t.} 	& \quad \sum_{i=1}^{n} x_i = 1, \ \sum_{i=1}^{n} y_i = 1\\
					& \quad x_i, y_i \in [0,1] \nonumber
	\text{,}
\end{align}
which is solved by \onehot{s} for $ \vec{x} $ and $ \vec{y} $ where $ x_i = 1 \neq y_i$ for a $ i \in \{1, \ ..., \ n\} $. Or to put it bluntly, the difference takes its highest values for all scenarios in which the vectors $ \vec{x} $ and $ \vec{y} $ are perpendicular and have a maximal euclidean norm, which means being a \onehot{}. This relation is present $ n $-times for the ground truth matrix and the learned one, implying that the maximal difference is $ \sqrt{2 \cdot n} $.

Finally, it is possible to define the metric on the set $ \mathcal{P} $ of $ n \times m $-probability matrices:
\begin{equation}
	d_A \colon \mathcal{P} \to [0,1], \qquad L \mapsto \frac{1}{\sqrt{2 \cdot n}}\Vert A - L \Vert_2
	\text{,}
\end{equation}
where $ A $ describes a fixed matrix in $ \mathcal{P} $. In the scope of this work the ground truth will play the role of $ A $. By $ d_A $, the learned matrix $ L $ is mapped to $ 0 $ if it matches the ground truth distribution perfectly and to $ 1 $ if the rows satisfy the conditions mentioned above.

% ======================================

%\subsection{Root-Mean-Square-Error}
%%\section{\gls{rmse}}
%The metric presented in the section beforehand is a generalization of the \gls{rmse}, which is also used for evaluation, especially in \secreff{sec: average approach}. It is defined as
%\begin{equation}
%	\mathrm{RSME} \colon \mathbb{R}^n \to \mathbb{R}_{\ge 0}, \qquad \vec{x} \mapsto  \sqrt{\sum_{i=0}^{n} \frac{x_i^2}{2}}
%	\text{.}
%\end{equation}
%The results of the models in \secreff{sec: average approach} are clearer to observe, so it is possible to analyze them more precisely \ie row-wise. Because the the objects of interest are also transition probability matrices, the connection between the lowest and highest values are preserved, if the difference between ground truth and learned row vector is plugged in. Hence, the range of the \gls{rmse} will be $ [0, 1] $. The equivalence between $ d $ and $ \mathrm{\gls{rmse}} $ follows for $ n = 1 $.


% ======================================


\chapter{Framework} \label{ch: framework}
After setting up the theoretical basis for the tools needed to start the experiments, the code framework will be introduced. In general, it is a supervised dense Neural Network written in \python{}~\cite{VanRossumEtAl09Python} with the help of \keras{}~\cite{chollet2015keras} and \numpy{}~\cite{harris2020array}. The process is divided into two phases: One training phase and one for visualization of the results. The goal was being capable to calculate a decent \srmat{} showing visible clusters in the \gls{mds}-plot. To retrieve the \gls{sr}, a Neural Network is trained, whose predictions serve as transition probability matrix $ T $. If not otherwise stated, a discount factor $ \gamma = 0.5 $ is used.

Different scenarios were tested, hence various models were configured having distinct features \eg some work with \onehot{s}, while other use word vectors or made up rules and datasets.

% ======================================

\section{First Model and Architecture} \label{subsec: first model and architecture}
The first class of networks augments the results in~\cite{Stöwer21MA}. P. Stöwer's models rely on predefined rules, like

\begin{itemize}
\phantomsection
\label{enum: rule set}
	\item \texttt{Adjective → Noun}
	\item \texttt{Verb → Adjective}
	\item \texttt{Personal Pronoun → Verb}
	\item \texttt{Question word → Personal Pronoun}
\end{itemize}
for building the dataset backed by a word database containing the corresponding information. Starting point is the \cognitiveroom{}, which consists of a list reflecting the whole data. The training data was crafted in accordance by randomly choosing respectively one of the four rules above and within the word class by chance an example. This information is used to initialize a \onehot{} and is done for input and output of the network.

The goal was to attain results on the behavior of the model if it is extended by more rules and words. One can imagine this type of model as a graph (\figref{\ref{fig: first model graph}}).

\begin{figure}
    \centering
    \includegraphics[scale=0.35]{Bilder/Graphen/first_model_graph2.png}
    \caption{The first two rules depicted as graph. In gray are corresponding \onehot{s} for the exemplary cognitive room \texttt{[blue, to run, desk]} denoted. The rules serve as edges and the word classes as vertices.}
    \label{fig: first model graph}
\end{figure}


% ======================================

\section{Word to word models} \label{sec: w2w models}
The more difficult challenge lies in unannotated texts without a paradigm for pairing words of an example data set. In this manner a language normally occurs and is learned by humans. A priori one has to expect results of poorer quality in comparison to the configuration of \secreff{subsec: first model and architecture} because they were tailored and \gls{nlp} comes always with uncertainties.

% - - - - - - - - - - - - - - - - - - -

\subsection{Data preparation} \label{sec: data preparation}
The data is extracted from two books, namely ``Gut gegen Nordwind'', written by Daniel Glattauer in German~\cite{Glattauer06GGW} and from Jostein Gaarder ``Sophie's World''~\cite{Gaarder96SW} in English. Two languages were chosen since German comes in general with a high degree of freedom in word order, in comparison English is more restrictive. This distinction may be important, since analyzing successive words is fundamental for this work. Because the books are available as \texttt{pdf}-file, the python module \pymupdf{}~\cite{pymupdf} is used to generate a simple \texttt{String} containing the whole text, which is afterwards parsed by \spacy{}~\cite{spacy2}. This is a powerful tool in the area of \gls{nlp} and some techniques are indispensable for further analysis, mainly
\begin{itemize}
	\item Tokenization: segmenting text into words, punctuations marks etc.
	\item \gls{pos}-Tagging\footnote{More information on \cite{udpostags}}: assigning word types to tokens, like verb or noun \label{item: pos tag}
	\item Lemmatization: assigning the base forms of words\footnote{For example, the lemma of “was” is “be”, and the lemma of “rats” is “rat”.}
	\item word2vec~\cite{MikolovEtAl13DRW, MikolovEtAl13EEW}: calculating a vector representation with real values of a word, in the following called \emph{word vector}.
\end{itemize}
%Lemmatization and Part-of-speech (POS) Tagging are used for bookkeeping and result evaluation.
%
Additionally, a mechanism was implemented to extract an exact number of words having equal sized foundations in both languages. The training data consists of word pairs in their occurring order, for instance the sentence
\begin{quote}
	Goethe remarked about Alexander von Humboldt to friends that he had never met anyone so versatile.\footnote{Sentence taken from \cite{Wulf16ION}} %\footcite{Wulf16ION}
\end{quote}
gets tokenized, lemmatized and coupled having
\begin{verbatim}
	("Goethe", "remark"), ("remark", "about"), ("about", "Alexander"), ...
\end{verbatim}
where the first component serves as input and the second one as supervised output.

Clearly, not the actual word is fed into the Neural Network, but numerical representations: either a \onehot{} or a word vector. To construct the former, the concept of the \cognitiveroom{} is applied by building a list containing all words of the text \ie one word resembles one state and states are encoded by place cells. To learn the transition probability matrix, as proclaimed at the beginning of the chapter, one has to transform the prediction of the Neural Network into a probability vector via division by its sum. This processing is done not during training because it is supervised via \onehot{s}.

% --------------------------------------

\subsection{\onehot{} approach} \label{subsubsec: onehot approach}
This configuration follows the principles of the first model (\secreff{subsec: first model and architecture}) by using \onehot{s} as input and output but there are no invented grammatical rules anymore. The training data is now directly related to concrete words and not to a word class. An illustration of the data structure is given in \figref{\ref{fig: text model graph ohe w2v}}.

% --------------------------------------

\subsection{Word vector approach} \label{subsubsec: word vector approach}
The Neural Network takes word vectors, a $ 300d $-vector of real numbers, as input and omits them during training. Word vectors are calculated by \spacy{}. To be precise, this step has two stages. Building the training data is easy because it is effortlessly possible to retrieve the real valued vector given a word. Since predictions aren't (and can't be) as accurate as the results \spacy{} computes, it is impossible to query a dictionary or database to reshape the exact word. For such situations, the module offers the option to retrieve a list with the $ n \in \mathbb{N} $ closest words. This list includes the $ n $ words whose vector representations have the smallest euclidean distance to the desired vector \ie the prediction.

In the next step, a check is performed whether the word is part of the book \ie the \cognitiveroom{}. If so, the euclidean distance is taken as entry in the Transition Probability Matrix $ T $, which will undergo a row-wise transformation to fit the criteria of a transition probability matrix. One disadvantage is an ambiguous prediction and therefore a less sharp $ T $. But in comparison to the \onehot{}, a word vector bears a lot more information, which hopefully can be exploited by the neural network. The data structure is shown in \figref{\ref{fig: text model graph ohe w2v}} next to the \onehot{} equivalent.

\begin{figure}
    \centering
        \includegraphics[width=\linewidth]{Bilder/Graphen/w2w_ohe_w2v.png}
    \caption{The word to word models can also be illustrated as graphs, though more complex. The lemmatized words of the text serve as vertices and the pairs mentioned before in \secreff{sec: data preparation} as edges. In green are word vectors and in blue \onehot{s} denoted. This cropped graph is generated by the bold passages of the text: ``\textbf{Alice sends Bob} a message. \textbf{Alice goes} to the grocery store. Peter \textbf{sent him} a letter. \textbf{Bob went} to his friend.''}
    \label{fig: text model graph ohe w2v}
\end{figure}

% --------------------------------------

%\subsection{Combined approach} \label{subsubsec: combined approach}
%The third model blends in both approaches by using the word vectors as input and a \onehot{} as output. It's goal is to tackle the imprecision aspects of the inevitable blur the word vector configuration experiences.


% ======================================

\section{Average approach} \label{subsubsec: average approach}
Because the models become enormous and  especially thus the evaluation difficult, this configuration aims to analyze the results on a rougher scale by taking averages of the predictions. While collecting the training data, the \postag{} of each one is saved and after training the cumulative outputs of a word class prediction are taken \ie all \onehot{s} of a \texttt{VERB} are mapped by the Neural Network, averaged into one vector and then checked for the most probable \postag{s} (\secreff{sec: average approach}).

\postag{s} were mentioned before in \secreff{sec: data preparation} and are just another term for word classes. In detail, a subset of the UniversalDependencies \postag{s}~\cite{udpostags} are used (\tabref{\ref{tab: ud pos tags}}) because this project provides a rich dataset, good documentation and its guidelines are implemented by \spacy{}.
\begin{table}
	\centering
	\caption[Listing of \postag{s}.]{List of relevant \postag{s} including examples and short definition.}
	\begin{tabular}{cl|cl}
		\toprule
		\postag{} 	  & Definition \& Example 		& \postag{} 		& Definition \& Example \\
		\midrule
		\texttt{ADJ}  & Adjective: educated, hot 	& \texttt{VERB} 	& Verb: to run, to drink \\
		\texttt{ADV}  & Adverb: easily, everywhere	& \texttt{DET} 		& Determiner: this, a, no \\
		\texttt{NOUN} & Noun: car, bottle 			& \texttt{PART}		& Particle: 's, not \\
		\texttt{AUX}  & Auxiliary: to have, should 	& \texttt{ADP}		& Adposition (Pre- \& Postpositions): in, on \\
		\texttt{PRON} & Pronoun: she, ours 			& \texttt{REST}		& \parbox{7cm}{Rest (Container for Conjunctions and additional residuals): and, if}\\
		\bottomrule
	\end{tabular}
	\label{tab: ud pos tags}
\end{table}
%In the next step the $ n $ indices with the highest values are checked for their word class, so if index $ i $ is one of the $ n $ highest and encodes the word \texttt{fish}, its word class \texttt{Noun} is counted. Finally, one has constructed a vector for each word class where each component resembles the probabilities after dividing by $ n $ for the following word class. An estimation of the \gls{sr} could look like the matrix in \eqref{eq: example average sr matrix} for the three word classes \texttt{Verb}, \texttt{Adjective} and \texttt{Noun} \ie the predicted transition probability for \texttt{Verb → Noun} is equal to $ 0.8 $, etc.
%\begin{equation} \label{eq: example average sr matrix}
%	\begin{pmatrix}
%		0 & 0.2 & 0.8 \\
%		0.1 & 0.3 & 0.6 \\
%		0.7 & 0.2 & 0.1
%	\end{pmatrix}
%\end{equation}



\section{Results}

% === Results – First model ===================================

\begin{frame}
\frametitle{Results – First model}
	\begin{itemize}
		\item<+-> Prediction works quite well i.e., the rules are recognizable e.g., {\huge \texttt{Adjective → Noun}}
		\item<2-> MDS plot shows clustered word classes 
	\end{itemize}
    \vspace*{-0.95cm}
	\begin{columns}
		% SPALTE 1
		\begin{column}{0.33\textwidth}
			\begin{figure}
			\centering
				\includegraphics[height=0.98\columnwidth]{Bilder/results_first_model/8Rules/plots/First Model + More Rules_100E_100BS_1L_1C/Transition_Probability_Matrix;_t=1,_DF=0.5.png}
			\end{figure}
            \begin{center}
                {\large Learned SR}
            \end{center}
		\end{column}
		% SPALTE 2
		\begin{column}{0.33\textwidth}
    			\begin{figure}
    				\centering
    					\includegraphics<2->[height=0.98\columnwidth]{Bilder/results_first_model/8Rules/plots/First Model + More Rules_100E_100BS_1L_1C/MDS_of_Transition_Probability_Matrix;_t=1,_DF=0.5.png}
    			\end{figure}
            \Os{2}{
                \begin{center}
                    {\large MDS plot}
                \end{center}
            }
		\end{column}
		% SPALTE 3
		\begin{column}{0.33\textwidth}
%			\vspace*{7mm}
			\begin{figure}
				\centering
					\includegraphics<3->[height=0.98\columnwidth]{Bilder/results_first_model/4Rules/plots/First Model + More Rules_100E_100BS_1L_1C/SR,_t=2,_DF=0.5.png}
			\end{figure}
        \Os{3}{
            \begin{center}
                {\large SR for $ t=2 $, less word classes}
            \end{center}
        }
		\end{column}
	\end{columns}
% --------------------------------------
\mynote{
\begin{itemize}
    \item[üü] Finally, I will present the results.
    \item[ü] Again, we'll begin talking about the First Model
	\item Works well because the SR performs best in these well defined scenarios
	\item[i] To avoid clutter only word classes are labeled, indeed one row corresponds to one word.
	\item The MDS plot shows clustered word classes, which also means learning was successful. Although not necessary for this type of model, it offers some visual feedback for configurations using a larger data set because their matrix can't be plotted
	\item LETZTES BILD: SR for t=2, it is possible to recognize the following states of a {\huge \texttt{question word}}, here they are {\huge \texttt{Personal Pronoun}} and {\huge \texttt{Verb}}
\end{itemize}
}
% SAINT MARTIN MEMMINGEN
\end{frame}

% === Results – Word to word models 1/2 ===================================

\begin{frame}
\frametitle{Results – Word to word models}
	\begin{itemize}
		\item<+-> Comparison with a ground truth/statistical assessment possible by a metric
	\end{itemize}
	\vspace*{-1.5cm}
	\begin{columns}
		% SPALTE 1
		\begin{column}{0.33\textwidth}
			\begin{figure}
				\centering
					\includegraphics[height=1.1\columnwidth]{Bilder/BspW2W/plots/OHE_OHE_500E_100BS_1L_1C_5P_30T_J/J_5pages_30T_words.png}
			\end{figure}
			\begin{center}
				{\large Ground truth, Transition probability matrix}
			\end{center}
		\end{column}
		% SPALTE 2
		\begin{column}{0.33\textwidth}
			\begin{figure}
				\centering
					\includegraphics[height=1.1\columnwidth]{Bilder/BspW2W/plots/OHE_OHE_500E_100BS_1L_1C_5P_30T_J/Transition_Probability_Matrix;_t=1,_DF=0.5.png}
			\end{figure}
			\begin{center}
                {\large Learned transition probability matrix}
			\end{center}
		\end{column}
		% SPALTE 3
		\begin{column}{0.33\textwidth}
            \vspace*{8.5mm}
			\begin{figure}
				\centering
					\includegraphics[height=1\columnwidth]{Bilder/BspW2W/plots/OHE_OHE_500E_100BS_1L_1C_5P_30T_J/MDS_of_Transition_Probability_Matrix;_t=1,_DF=0.5.png}
			\end{figure}
			\begin{center}
                {\large Learned MDS}
			\end{center}
		\end{column}
	\end{columns}
% --------------------------------------
\mynote{
\begin{itemize}
    \item[ü] word to word models show a different outcome.
    \item It is possible to compare the results to a ground truth/statistical assessment, hence the Metric $ d_A $ comes into play
	\item[i] On the left (LINKES BILD) is a ground truth depicted and next to it the predictions of the network. Although there is a resemblance visible it has to be assessed cautiously because a tiny data set was used for illustration purposes only to convey an intuition for the results and the procedure.
	\item[i] The MDS is displayed because matrices won't provide visual feedback anymore and as seen before sufficient learning is also visible in the cluster plot.
\end{itemize}
}
\end{frame}

% --- Results – Word to word modeles 2/2 -----------------------------------

\begin{frame}
	\frametitle{Results – Word to word models}
	\begin{itemize}
		\item<1-> It is possible to compare the results to a ground truth/statistical assessment $\Longrightarrow $ Metric $ d_A $
		\item<2-> Surprisingly \onehot{s} outperform word vectors i.e., word vectors are just bad
		\item<3-> German or english doesn't make that much of a difference
	\end{itemize}
	\begin{columns}
		% SPALTE 1
		\begin{column}{0.4\textwidth}
    \Os{1}{
			\vspace*{3mm}
			\begin{table}
%				\caption{Configurations with metric w.r.t. ground truth}
				\begin{tabular}{ll}
					\toprule
					Version					& Metric \\
					\midrule
					german, \onehot{} 		& $ 0.08 $	\\% 0.08200
					german, word vector		& $ 0.74 $	\\% 0.74139
					english, \onehot{}		& $ 0.10 $	\\% 0.10021
					english, word vector	& $ 0.78 $	\\% 0.77522
					\bottomrule
				\end{tabular}
			\end{table}
			\begin{center}
				{\large Configurations \& metric w.r.t. ground truth}
			\end{center}
            \vspace{3.2cm}\hspace*{5mm}\parbox{0.9\columnwidth}{{\normalsize \notsoimportant{If you want to know more about the metric, you can ask after the talk}}}
    }
		\end{column}
		% SPALTE 2
		\begin{column}{0.3\textwidth}
			\vspace*{-1cm}
			\begin{figure} % ohe de
				\centering
					\includegraphics<4->[height=0.55\textheight]{Bilder/W2W/OHE_OHE_5000E_100BS_1L_1C_200P_1500T_D/MDS_of_Transition_Probability_Matrix;_t=1,_DF=0.5.png}
			\end{figure}
    \Os{4}{
			\begin{center}
				{\large MDS of german, \onehot{}}
			\end{center}
    }
		\end{column}
		% SPALTE 3
		\begin{column}{0.3\textwidth}
			\vspace*{-1cm}
			\begin{figure} % w2v de
				\centering
					\includegraphics<4->[height=0.55\textheight]{Bilder/W2W/W2V_W2V_5000E_100BS_1L_1C_200P_1500T_D/MDS_of_Transition_Probability_Matrix;_t=1,_DF=0.5.png}
			\end{figure}
    \Os{4}{
			\begin{center}
				{\large MDS of german, word vectors}
			\end{center}
    }
		\end{column}
	\end{columns}
\mynote{
	\begin{itemize}
		\item In the table we find all four configurations comprising of \onehot{s} and word vector each ran with german and english.
        \begin{itemize}
            \item[i] The metric maps matrices close to $ 0 $ if the equal the ground truth and to $ 1 $ if not
        \end{itemize}
		\item Surprisingly \onehot{s} perform better than word vectors. Since they encode a word with more than two numbers, this was nothing to reckon with.
		\item We expected that english performs better than german due to the more static word order which was a fallacy
		\item In the image on the left is the MDS plot of the SR using \onehot{s} illustrated and on the right using word vectors. In both cases the hoped clusters didn't form and the configuration on the right seems to be omit a degenerated output at all. 
	\end{itemize}
}
\end{frame}

% === Results – Averaging models 1/3 ===================================

\begin{frame}{Results – Averaging models}
    \begin{itemize}
        \item<1-> Outcome of the plain vector models wasn't satisfying (as seen in the MDS plots), so averaging was established
    \item<2-> Results were indeed exploitable i.e., word class transition probabilities are partially reflected very accurately
    \end{itemize}
    \begin{columns}
        % SPALTE 1 ADP barplot
        \begin{column}{0.5\textwidth}
            \vspace*{-1cm}
            \begin{figure}
                \centering
                    \includegraphics<2->[width=0.66\textheight]{Bilder/Barplots/Avg_OHE_OHE_5000E_100BS_1L_1C_200P_1500T_D/Combined_Barplot_ADJ_F.png}
            \end{figure}
        \end{column}
        % SPALTE 2 ADJ barplot
        \begin{column}{0.5\textwidth}
            \vspace*{-1cm}
            \begin{figure}
                \centering
                    \includegraphics<2->[height=0.66\textheight]{Bilder/Barplots/Avg_OHE_OHE_5000E_100BS_1L_1C_200P_1500T_D/Combined_Barplot_ADP_S.png}
            \end{figure}
        \end{column}
    \end{columns}
% --------------------------------------
\mynote{
\begin{itemize}
    \item[ü] After all, coming to the Average Approach
    \item Because the outcome of the plain vector models wasn't satisfying (as illustarted in the MDS plot beforehand), an averaging approach was developed. Based on the predictions shown beforehand, transition probabilities between word classes were calculated
    \item The results are indeed exploitable especially using \onehot{s}. There, word class transitions are partially reflected very accurately as we can see in both bar plots
    \item[i] In both figures the learned probabilities and the ground truth probabilities are plotted, in blue and green respectively. The title is referring to the first word class, whereas the x axis shows the particular successor. Both diagrams show good accuracy for the word classes. 
\end{itemize}
}
\end{frame}

% --- Results – Averaging models 2/? -----------------------------------

\begin{frame}{Results – Averaging models}
    \begin{itemize}
        \item Matrices are $ 10 \times 10 $, so we display them
    \end{itemize}
    \vspace*{-1.5cm}
    \begin{columns}
        % SPALTE 1
        \begin{column}{0.33\textwidth}
            \begin{figure}
                \centering
                \includegraphics[width=1.1\columnwidth]{Bilder/average_models/ground_truths/D_200pages_1500T_tags.png}
            \end{figure}
            \begin{center}
                {\large ground truth (german)}
            \end{center}
        \end{column}
        % SPALTE 2
        \begin{column}{0.33\textwidth}
            \begin{figure}
                \centering
                \includegraphics[width=1.1\columnwidth]{Bilder/average_models/Avg_OHE_OHE_5000E_100BS_1L_1C_200P_1500T_D/Transition_Probability_Matrix;_t=1,_DF=0.5.png}
            \end{figure}
            \begin{center}
                {\large german, \onehot{}} %:\\$ \mu = 7.3$, $ \sigma = 2.0 $
            \end{center}
        \end{column}
        % SPALTE 3
        \begin{column}{0.33\textwidth}
            \begin{figure}
                \centering
                \includegraphics[width=1.1\columnwidth]{Bilder/average_models/Avg_W2V_W2V_5000E_100BS_1L_1C_200P_1500T_D/Transition_Probability_Matrix;_t=1,_DF=0.5.png}
            \end{figure}
            \begin{center}
                {\large german, word vector} %:\\$ \mu = 14.0$, $ \sigma = 2.1 $
            \end{center}
        \end{column}
    \end{columns}
% --------------------------------------
\mynote{
\begin{itemize}
    \item Matrices are $ 10 \times 10 $, so we display them
    \item[i] On the left there is the ground truth of the book illustrated, in the middle a \onehot{} model and on the right the word vector equivalent. The latter shows no learning at all, whereas similarities to the ground truth are recognizable in the middle.
    \item[i] The picture is similar for the english book
\end{itemize}
}
\end{frame}


% --- Results – Averaging models 2/3 -----------------------------------

\begin{frame}{Results – Averaging models}
    \begin{itemize}
        \item<+-> Accuracy of these models is measured by mean and standard deviation:
    \end{itemize}
    \begin{table}
        \centering
%        \caption{Averaged means of the difference between prediction and ground truth. As expected, the word vector versions have higher scores than the \onehot{} counterpart. These values can't be used for a comparison with \tabref{\ref{tab: text model versions and metrics}} because the underlying metrics are different.}
        \begin{tabular}{lrc}
            \toprule
            Version					& Mean $ \mu $		& Standard deviation $ \sigma $ \\
            \midrule
            german, \onehot{} 		& $ 7.3 $	& $ 2.0 $ \\% 0.08200
            german, word vector		& $ 14.0 $	& $ 2.1 $ \\% 0.74139
            english, \onehot{}		& $ 8.1 $	& $ 3.3 $ \\% 0.10021
            english, word vector	& $ 10.2 $	& $ 3.6 $ \\% 0.77522
            \bottomrule
        \end{tabular}
        \label{tab: avg model versions and metrics}
    \end{table}
    \hfill\notsoimportant{Mean and standard deviation in $ 10^{-2} $}
    \begin{itemize}
%        \item<+-> Again \onehot{s} outperform word vectors
        \item<+-> Sadly, the outcome of word vector models is quite bad again % (word vectors might be closer to real input signals of the hippocampus)
        \item<+-> But the \onehot{s} seem to grasp the grammatical structure (bar and matrix plot)
    \end{itemize}
% --------------------------------------
\mynote{
\begin{itemize}
    \item The accuracy of these models is measured by mean and standard deviation 	
    \item Sadly, the outcome of word vector models is not worth mentioning it again. This is in particular unsatisfactory because word vectors might be closer to real signals.
    \item But \onehot{} approaches seem to grasp the grammatical structure, which is justified by the bar and matrix plots   
    \item[x] FALLS JEMAND FRAGT Anderes Maß, es hier explizit die verschiedenen Zeilen mit der Ground truth verglichen wurden, Außerdem hat man weniger Spalten bzw. Zustände im Allgemeinen vorliegen, sodass man bspw. mit der Angabe der Standardabweichung auch etwas anfangen kann
    \item[x] FALLS JEMAND FRAGT
    \begin{itemize}
        \item[x] „Mean“ bedeutet: GT - SR, then row-wise mean, finally mean of means
        \item[x] „Std. deviation“ GT - SR, then row-wise std. dev., finally std. dev of std. devs.
    \end{itemize}
\end{itemize}
}
\end{frame}

% === CONCLUSION ===================================

\section{Conclusion}
\begin{frame}{Conclusion}
    \begin{itemize}
        \item<+-> By far most of the time was consumed by finding proper values, sadly with bad luck
        \item<+-> Plenty of configurations didn't improve the results or were worse. Two of them were
        \begin{itemize}
            \item<+-> Multiple hidden layers
            \item<+-> Predicting only most frequent words
        \end{itemize}
        \item<+-> Due to the lack of valid data from real experiments interpretation regarding our daily life is difficult
        \item<+-> Performance of word vectors disappointing, which is a drawback because they might be closer to actual signals
        \item<+-> Some learning does happen (Average approach)
    \end{itemize}
    \begin{figure}
        \centering
            \includegraphics<7->[height=0.4\textheight]{Bilder/average_models/ground_truths/D_200pages_1500T_tags.png}
            \includegraphics<7->[height=0.4\textheight]{Bilder/average_models/Avg_OHE_OHE_5000E_100BS_1L_1C_200P_1500T_D/Transition_Probability_Matrix;_t=1,_DF=0.5.png}
    \end{figure}

% --- NOTIZEN -----------------------------------
\mynote{
    \begin{itemize}
        \item[ü] Coming to the Conclusion
        \item By far most of the time was consumed by finding well functioning values.
        \item Plenty of configurations didn't improve the results or were worse. Although it took much space during the months the dead ends aren't illustrated in the presentation but in the thesis. Two of them were
        \begin{itemize}
            \item Multiple hidden layers
            \item Predicting only most frequent words
        \end{itemize}
        \item Due to the lack of valid data from real experiments interpretation regarding our daily life is difficult
        \item Unfortunately, performance of word vectors is in general disappointing, which is a drawback because they might be closer to actual signals
        \item Nevertheless, some learning does happen and paths can be reconstructed (Average approach) ENDE DES VORTRAGS
    \end{itemize}
}
\end{frame}

% === REFERENCES ===================================

\section*{References}
\begin{frame}[allowframebreaks,noframenumbering]{References}
	\printbibliography
\end{frame}
%}

\finalframe{Thank you\\for your attention!}

\end{document}
